

@article{S
	title = {Proximity-{Induced} {Ferromagnetism} in {Graphene} {Revealed} by the {Anomalous} {Hall} {Effect}},
	volume = {114},
	url = {https://link.aps.org/doi/10.1103/PhysRevLett.114.016603},
	doi = {10.1103/PhysRevLett.114.016603},
	abstract = {We demonstrate the anomalous Hall effect (AHE) in single-layer graphene exchange coupled to an atomically flat yttrium iron garnet (YIG) ferromagnetic thin film. The anomalous Hall conductance has magnitude of ∼0.09(2e2/h) at low temperatures and is measurable up to ∼300 K. Our observations indicate not only proximity-induced ferromagnetism in graphene/YIG with a large exchange interaction, but also enhanced spin-orbit coupling that is believed to be inherently weak in ideal graphene. The proximity-induced ferromagnetic order in graphene can lead to novel transport phenomena such as the quantized AHE which are potentially useful for spintronics.},
	number = {1},
	urldate = {2020-05-05},
	journal = {Physical Review Letters},
	author = {Wang, Zhiyong and Tang, Chi and Sachs, Raymond and Barlas, Yafis and Shi, Jing},
	month = jan,
	year = {2015},
	note = {Publisher: American Physical Society},
	pages = {016603},
	file = {Accepted Version:/Users/sweetie/Zotero/storage/YP2MPXUW/Wang et al. - 2015 - Proximity-Induced Ferromagnetism in Graphene Revea.pdf:application/pdf;APS Snapshot:/Users/sweetie/Zotero/storage/3F2YGZMR/PhysRevLett.114.html:text/html}
}

@article{leutenantsmeyer_proximity_2016,
	title = {Proximity induced room temperature ferromagnetism in graphene probed with spin currents},
	volume = {4},
	issn = {2053-1583},
	url = {https://doi.org/10.1088%2F2053-1583%2F4%2F1%2F014001},
	doi = {10.1088/2053-1583/4/1/014001},
	abstract = {We present a direct measurement of the exchange interaction in room temperature ferromagnetic graphene. We study the spin transport in exfoliated graphene on an yttrium–iron–garnet substrate where the observed spin precession clearly indicates the presence and strength of an exchange field that is an unambiguous evidence of induced ferromagnetism. We describe the results with a modified Bloch diffusion equation and extract an average exchange field of the order of 0.2 T. Further, we demonstrate that a proximity induced 2D ferromagnet can efficiently modulate a spin current by controlling the direction of the exchange field. These findings can create a building block for magnetic-gate tuneable spin transport in one-atom-thick spintronic devices.},
	language = {en},
	number = {1},
	urldate = {2020-05-05},
	journal = {2D Materials},
	author = {Leutenantsmeyer, Johannes Christian and Kaverzin, Alexey A. and Wojtaszek, Magdalena and Wees, Bart J. van},
	month = nov,
	year = {2016},
	note = {Publisher: IOP Publishing},
	pages = {014001},
	file = {IOP Full Text PDF:/Users/sweetie/Zotero/storage/D8HS76ZD/Leutenantsmeyer et al. - 2016 - Proximity induced room temperature ferromagnetism .pdf:application/pdf}
}
@article{han_graphene_2014,
	title = {Graphene spintronics},
	volume = {9},
	copyright = {2014 Nature Publishing Group, a division of Macmillan Publishers Limited. All Rights Reserved.},
	issn = {1748-3395},
	url = {https://www.nature.com/articles/nnano.2014.214},
	doi = {10.1038/nnano.2014.214},
	abstract = {Spin-dependent phenomena and applications in graphene and other 2D materials are discussed in this Review.},
	language = {en},
	number = {10},
	urldate = {2020-05-05},
	journal = {Nature Nanotechnology},
	author = {Han, Wei and Kawakami, Roland K. and Gmitra, Martin and Fabian, Jaroslav},
	month = oct,
	year = {2014},
	note = {Number: 10
Publisher: Nature Publishing Group},
	pages = {794--807},
	file = {Full Text PDF:/Users/sweetie/Zotero/storage/ZTDR6VJN/Han et al. - 2014 - Graphene spintronics.pdf:application/pdf;Snapshot:/Users/sweetie/Zotero/storage/VKHQYFPU/nnano.2014.html:text/html}
}

@article{ugeda_missing_2010,
	title = {Missing {Atom} as a {Source} of {Carbon} {Magnetism}},
	volume = {104},
	url = {https://link.aps.org/doi/10.1103/PhysRevLett.104.096804},
	doi = {10.1103/PhysRevLett.104.096804},
	abstract = {Atomic vacancies have a strong impact in the mechanical, electronic, and magnetic properties of graphenelike materials. By artificially generating isolated vacancies on a graphite surface and measuring their local density of states on the atomic scale, we have shown how single vacancies modify the electronic properties of this graphenelike system. Our scanning tunneling microscopy experiments, complemented by tight-binding calculations, reveal the presence of a sharp electronic resonance at the Fermi energy around each single graphite vacancy, which can be associated with the formation of local magnetic moments and implies a dramatic reduction of the charge carriers’ mobility. While vacancies in single layer graphene lead to magnetic couplings of arbitrary sign, our results show the possibility of inducing a macroscopic ferrimagnetic state in multilayered graphene just by randomly removing single C atoms.},
	number = {9},
	urldate = {2020-05-05},
	journal = {Physical Review Letters},
	author = {Ugeda, M. M. and Brihuega, I. and Guinea, F. and Gómez-Rodríguez, J. M.},
	month = mar,
	year = {2010},
	note = {Publisher: American Physical Society},
	pages = {096804},
	file = {APS Snapshot:/Users/sweetie/Zotero/storage/D7MFFBFZ/PhysRevLett.104.html:text/html;Submitted Version:/Users/sweetie/Zotero/storage/VWWDTRNM/Ugeda et al. - 2010 - Missing Atom as a Source of Carbon Magnetism.pdf:application/pdf}
}

@article{han_perspectives_2016,
	title = {Perspectives for spintronics in {2D} materials},
	volume = {4},
	url = {https://aip.scitation.org/doi/10.1063/1.4941712},
	doi = {10.1063/1.4941712},
	abstract = {The past decade has been especially creative for spintronics since the (re)discovery of various two dimensional (2D) materials. Due to the unusual physical characteristics, 2D materials have provided new platforms to probe the spin interaction with other degrees of freedom for electrons, as well as to be used for novel spintronics applications. This review briefly presents the most important recent and ongoing research for spintronics in 2D materials.},
	number = {3},
	urldate = {2020-05-05},
	journal = {APL Materials},
	author = {Han, Wei},
	month = feb,
	year = {2016},
	note = {Publisher: American Institute of Physics},
	pages = {032401},
	file = {Full Text PDF:/Users/sweetie/Zotero/storage/F9PHGN24/Han - 2016 - Perspectives for spintronics in 2D materials.pdf:application/pdf;Snapshot:/Users/sweetie/Zotero/storage/SWYNRZW6/1.html:text/html}
}

@article{gonzalez-herrero_atomic-scale_2016,
	title = {Atomic-scale control of graphene magnetism by using hydrogen atoms},
	volume = {352},
	issn = {0036-8075, 1095-9203},
	url = {https://www.sciencemag.org/lookup/doi/10.1126/science.aad8038},
	doi = {10.1126/science.aad8038},
	language = {en},
	number = {6284},
	urldate = {2020-05-05},
	journal = {Science},
	author = {Gonzalez-Herrero, H. and Gomez-Rodriguez, J. M. and Mallet, P. and Moaied, M. and Palacios, J. J. and Salgado, C. and Ugeda, M. M. and Veuillen, J.-Y. and Yndurain, F. and Brihuega, I.},
	month = apr,
	year = {2016},
	pages = {437--441},
	file = {Gonzalez-Herrero et al. - 2016 - Atomic-scale control of graphene magnetism by usin.pdf:/Users/sweetie/Zotero/storage/D8V9UGIG/Gonzalez-Herrero et al. - 2016 - Atomic-scale control of graphene magnetism by usin.pdf:application/pdf}
}
@article{dietl_dilute_2014,
	title = {Dilute ferromagnetic semiconductors: {Physics} and spintronic structures},
	volume = {86},
	shorttitle = {Dilute ferromagnetic semiconductors},
	url = {https://link.aps.org/doi/10.1103/RevModPhys.86.187},
	doi = {10.1103/RevModPhys.86.187},
	abstract = {This review compiles results of experimental and theoretical studies on thin films and quantum structures of semiconductors with randomly distributed Mn ions, which exhibit spintronic functionalities associated with collective ferromagnetic spin ordering. Properties of p-type Mn-containing III-V as well as II-VI, IV-VI, V2−VI3, I-II-V, and elemental group IV semiconductors are described, paying particular attention to the most thoroughly investigated system (Ga,Mn)As that supports the hole-mediated ferromagnetic order up to 190 K for the net concentration of Mn spins below 10\%. Multilayer structures showing efficient spin injection and spin-related magnetotransport properties as well as enabling magnetization manipulation by strain, light, electric fields, and spin currents are presented together with their impact on metal spintronics. The challenging interplay between magnetic and electronic properties in topologically trivial and nontrivial systems is described, emphasizing the entangled roles of disorder and correlation at the carrier localization boundary. Finally, the case of dilute magnetic insulators is considered, such as (Ga,Mn)N, where low-temperature spin ordering is driven by short-ranged superexchange that is ferromagnetic for certain charge states of magnetic impurities.},
	number = {1},
	urldate = {2020-05-05},
	journal = {Reviews of Modern Physics},
	author = {Dietl, Tomasz and Ohno, Hideo},
	month = mar,
	year = {2014},
	note = {Publisher: American Physical Society},
	pages = {187--251},
	file = {APS Snapshot:/Users/sweetie/Zotero/storage/WZ9HZTWE/RevModPhys.86.html:text/html}
}

@incollection{liu_chapter_2020,
	series = {Materials {Today}},
	title = {Chapter 1 - {Introduction} to spintronics and {2D} materials},
	isbn = {978-0-08-102154-5},
	url = {http://www.sciencedirect.com/science/article/pii/B9780081021545000011},
	abstract = {Spin-based technologies, in the form of magnetic compasses, have existed for thousands of years. However, it is only in the last century that the concept of spin, as a pure quantum phenomenon, was established, paving the way for new applications exploiting its quantum mechanical properties. This chapter introduces the underlying mechanisms behind spin ordering in magnetic materials, particularly in two-dimensional (2D) limit, and reviews historical milestones in the development of “spintronics” and electronic devices utilizing spin. Initially studied for fundamental physical interest, spintronic applications quickly came to prominence following the discovery of giant magnetoresistance and the rediscovery of tunneling magnetoresistance in metallic architectures. The desire to integrate these phenomena with the existing semiconductor-based technologies used in the computing industry has inspired intensive study of spintronics within semiconductor materials. Notable developments include the spin field-effect transistor and nonlocal device geometries. Following the discovery of novel electronic properties of graphene, there has been a drive to harness similar properties–with the addition of spin-control–in van der Waals or 2D materials, resulting in several families of 2D magnetic materials known today.},
	language = {en},
	urldate = {2020-05-05},
	booktitle = {Spintronic {2D} {Materials}},
	publisher = {Elsevier},
	author = {Liu, Wenqing and Bryan, Matthew T. and Xu, Yongbing},
	editor = {Liu, Wenqing and Xu, Yongbing},
	month = jan,
	year = {2020},
	doi = {10.1016/B978-0-08-102154-5.00001-1},
	keywords = {2D magnetism, 2D materials, Spintronics},
	pages = {1--24},
	file = {ScienceDirect Snapshot:/Users/sweetie/Zotero/storage/AU39K75P/B9780081021545000011.html:text/html}
}
@book{tsymbal_spintronics_2019,
	title = {Spintronics {Handbook}, {Second} {Edition}: {Spin} {Transport} and {Magnetism}: {Volume} {Three}: {Nanoscale} {Spintronics} and {Applications}},
	isbn = {978-0-429-80525-7},
	shorttitle = {Spintronics {Handbook}, {Second} {Edition}},
	abstract = {Spintronics Handbook, Second Edition offers an update on the single most comprehensive survey of the two intertwined fields of spintronics and magnetism, covering the diverse array of materials and structures, including silicon, organic semiconductors, carbon nanotubes, graphene, and engineered nanostructures. It focuses on seminal pioneering work, together with the latest in cutting-edge advances, notably extended discussion of two-dimensional materials beyond graphene, topological insulators, skyrmions, and molecular spintronics. The main sections cover physical phenomena, spin-dependent tunneling, control of spin and magnetism in semiconductors, and spin-based applications. Features:   Presents the most comprehensive reference text for the overlapping fields of spintronics (spin transport) andmagnetism. Covers the full spectrum of materials and structures, from silicon and organic semiconductors to carbon nanotubes, graphene, and engineered nanostructures. Extends coverage of two-dimensional materials beyond graphene, including molybdenum disulfide and study of their spin relaxation mechanisms Includes new dedicated chapters on cutting-edge topics such as spin-orbit torques, topological insulators, half metals, complex oxide materials and skyrmions. Discusses important emerging areas of spintronics with superconductors, spin-wave spintronics, benchmarking of spintronics devices, and theory and experimental approaches to molecular spintronics. Evgeny Tsymbal's research is focused on computational materials science aiming at the understanding of fundamental properties of advanced ferromagnetic and ferroelectric nanostructures and materials relevant to nanoelectronics and spintronics. He is a George Holmes University Distinguished Professor at the Department of Physics and Astronomy of the University of Nebraska-Lincoln (UNL), Director of the UNL’s Materials Research Science and Engineering Center (MRSEC), and Director of the multi-institutional Center for NanoFerroic Devices (CNFD).  Igor Žutić received his Ph.D. in theoretical physics at the University of Minnesota. His work spans a range of topics from high-temperature superconductors and ferromagnetism that can get stronger as the temperature is increased, to prediction of various spin-based devices. He is a recipient of 2006 National Science Foundation CAREER Award, 2005 National Research Council/American Society for Engineering Education Postdoctoral Research Award, and the National Research Council Fellowship (2003-2005). His research is supported by the National Science Foundation, the Office of Naval Research, the Department of Energy, and the Airforce Office of Scientific Research.},
	language = {en},
	publisher = {CRC Press},
	author = {Tsymbal, Evgeny Y. and Žutić, Igor},
	month = jun,
	year = {2019},
	keywords = {Science / Mechanics / Thermodynamics, Science / Physics / Condensed Matter, Science / Physics / General, Technology \& Engineering / Electronics / General, Technology \& Engineering / Materials Science / Electronic Materials, Technology \& Engineering / Materials Science / General}
}

@article{novoselov_two-dimensional_2005,
	title = {Two-dimensional atomic crystals},
	volume = {102},
	copyright = {Copyright © 2005, The National Academy of Sciences},
	issn = {0027-8424, 1091-6490},
	url = {https://www.pnas.org/content/102/30/10451},
	doi = {10.1073/pnas.0502848102},
	abstract = {We report free-standing atomic crystals that are strictly 2D and can be viewed as individual atomic planes pulled out of bulk crystals or as unrolled single-wall nanotubes. By using micromechanical cleavage, we have prepared and studied a variety of 2D crystals including single layers of boron nitride, graphite, several dichalcogenides, and complex oxides. These atomically thin sheets (essentially gigantic 2D molecules unprotected from the immediate environment) are stable under ambient conditions, exhibit high crystal quality, and are continuous on a macroscopic scale.},
	language = {en},
	number = {30},
	urldate = {2020-05-05},
	journal = {Proceedings of the National Academy of Sciences},
	author = {Novoselov, K. S. and Jiang, D. and Schedin, F. and Booth, T. J. and Khotkevich, V. V. and Morozov, S. V. and Geim, A. K.},
	month = jul,
	year = {2005},
	pmid = {16027370},
	note = {Publisher: National Academy of Sciences
Section: Physical Sciences},
	keywords = {graphene, layered material},
	pages = {10451--10453},
	file = {Full Text PDF:/Users/sweetie/Zotero/storage/H82Q2ZPK/Novoselov et al. - 2005 - Two-dimensional atomic crystals.pdf:application/pdf;Snapshot:/Users/sweetie/Zotero/storage/EVVYA3SH/10451.html:text/html}
}

@article{radisavljevic_single-layer_2011,
	title = {Single-layer {MoS} 2 transistors},
	volume = {6},
	copyright = {2011 Nature Publishing Group},
	issn = {1748-3395},
	url = {https://www.nature.com/articles/nnano.2010.279},
	doi = {10.1038/nnano.2010.279},
	abstract = {The large bandgap of a single layer of molybdenum disulphide can be exploited to construct transistors with high on/off ratios and high mobilities.},
	language = {en},
	number = {3},
	urldate = {2020-05-05},
	journal = {Nature Nanotechnology},
	author = {Radisavljevic, B. and Radenovic, A. and Brivio, J. and Giacometti, V. and Kis, A.},
	month = mar,
	year = {2011},
	note = {Number: 3
Publisher: Nature Publishing Group},
	pages = {147--150},
	file = {Full Text:/Users/sweetie/Zotero/storage/94BDGRKH/Radisavljevic et al. - 2011 - Single-layer MoS 2 transistors.pdf:application/pdf;Snapshot:/Users/sweetie/Zotero/storage/8W6484VS/nnano.2010.html:text/html}
}

@article{mak_atomically_2010,
	title = {Atomically {Thin} \$\{{\textbackslash}mathrm\{{MoS}\}\}\_\{2\}\$: {A} {New} {Direct}-{Gap} {Semiconductor}},
	volume = {105},
	shorttitle = {Atomically {Thin} \$\{{\textbackslash}mathrm\{{MoS}\}\}\_\{2\}\$},
	url = {https://link.aps.org/doi/10.1103/PhysRevLett.105.136805},
	doi = {10.1103/PhysRevLett.105.136805},
	abstract = {The electronic properties of ultrathin crystals of molybdenum disulfide consisting of N=1,2,…,6 S-Mo-S monolayers have been investigated by optical spectroscopy. Through characterization by absorption, photoluminescence, and photoconductivity spectroscopy, we trace the effect of quantum confinement on the material’s electronic structure. With decreasing thickness, the indirect band gap, which lies below the direct gap in the bulk material, shifts upwards in energy by more than 0.6 eV. This leads to a crossover to a direct-gap material in the limit of the single monolayer. Unlike the bulk material, the MoS2 monolayer emits light strongly. The freestanding monolayer exhibits an increase in luminescence quantum efficiency by more than a factor of 104 compared with the bulk material.},
	number = {13},
	urldate = {2020-05-05},
	journal = {Physical Review Letters},
	author = {Mak, Kin Fai and Lee, Changgu and Hone, James and Shan, Jie and Heinz, Tony F.},
	month = sep,
	year = {2010},
	note = {Publisher: American Physical Society},
	pages = {136805},
	file = {APS Snapshot:/Users/sweetie/Zotero/storage/H3H49MMD/PhysRevLett.105.html:text/html;Submitted Version:/Users/sweetie/Zotero/storage/53N3S45C/Mak et al. - 2010 - Atomically Thin \$ mathrm MoS _ 2 \$ A New Direct.pdf:application/pdf}
}

@article{xiao_coupled_2012,
	title = {Coupled {Spin} and {Valley} {Physics} in {Monolayers} of \$\{{\textbackslash}mathrm\{{MoS}\}\}\_\{2\}\$ and {Other} {Group}-{VI} {Dichalcogenides}},
	volume = {108},
	url = {https://link.aps.org/doi/10.1103/PhysRevLett.108.196802},
	doi = {10.1103/PhysRevLett.108.196802},
	abstract = {We show that inversion symmetry breaking together with spin-orbit coupling leads to coupled spin and valley physics in monolayers of MoS2 and other group-VI dichalcogenides, making possible controls of spin and valley in these 2D materials. The spin-valley coupling at the valence-band edges suppresses spin and valley relaxation, as flip of each index alone is forbidden by the valley-contrasting spin splitting. Valley Hall and spin Hall effects coexist in both electron-doped and hole-doped systems. Optical interband transitions have frequency-dependent polarization selection rules which allow selective photoexcitation of carriers with various combination of valley and spin indices. Photoinduced spin Hall and valley Hall effects can generate long lived spin and valley accumulations on sample boundaries. The physics discussed here provides a route towards the integration of valleytronics and spintronics in multivalley materials with strong spin-orbit coupling and inversion symmetry breaking.},
	number = {19},
	urldate = {2020-05-05},
	journal = {Physical Review Letters},
	author = {Xiao, Di and Liu, Gui-Bin and Feng, Wanxiang and Xu, Xiaodong and Yao, Wang},
	month = may,
	year = {2012},
	note = {Publisher: American Physical Society},
	pages = {196802},
	file = {Accepted Version:/Users/sweetie/Zotero/storage/R6I9R9RP/Xiao et al. - 2012 - Coupled Spin and Valley Physics in Monolayers of \$.pdf:application/pdf;APS Snapshot:/Users/sweetie/Zotero/storage/H2WLR7CG/PhysRevLett.108.html:text/html}
}

@article{zeng_valley_2012,
	title = {Valley polarization in {MoS} 2 monolayers by optical pumping},
	volume = {7},
	copyright = {2012 Nature Publishing Group},
	issn = {1748-3395},
	url = {https://www.nature.com/articles/nnano.2012.95},
	doi = {10.1038/nnano.2012.95},
	abstract = {Circularly polarized light has been used to achieve a valley polarization of 30\% in single-layer molybdenum disulphide.},
	language = {en},
	number = {8},
	urldate = {2020-05-05},
	journal = {Nature Nanotechnology},
	author = {Zeng, Hualing and Dai, Junfeng and Yao, Wang and Xiao, Di and Cui, Xiaodong},
	month = aug,
	year = {2012},
	note = {Number: 8
Publisher: Nature Publishing Group},
	pages = {490--493},
	file = {Snapshot:/Users/sweetie/Zotero/storage/H6YPUF7B/nnano.2012.html:text/html;Submitted Version:/Users/sweetie/Zotero/storage/WMQFR4JW/Zeng et al. - 2012 - Valley polarization in MoS 2 monolayers by optical.pdf:application/pdf}
}

@article{butler_progress_2013,
	title = {Progress, {Challenges}, and {Opportunities} in {Two}-{Dimensional} {Materials} {Beyond} {Graphene}},
	volume = {7},
	issn = {1936-0851},
	url = {https://doi.org/10.1021/nn400280c},
	doi = {10.1021/nn400280c},
	abstract = {Graphene’s success has shown that it is possible to create stable, single and few-atom-thick layers of van der Waals materials, and also that these materials can exhibit fascinating and technologically useful properties. Here we review the state-of-the-art of 2D materials beyond graphene. Initially, we will outline the different chemical classes of 2D materials and discuss the various strategies to prepare single-layer, few-layer, and multilayer assembly materials in solution, on substrates, and on the wafer scale. Additionally, we present an experimental guide for identifying and characterizing single-layer-thick materials, as well as outlining emerging techniques that yield both local and global information. We describe the differences that occur in the electronic structure between the bulk and the single layer and discuss various methods of tuning their electronic properties by manipulating the surface. Finally, we highlight the properties and advantages of single-, few-, and many-layer 2D materials in field-effect transistors, spin- and valley-tronics, thermoelectrics, and topological insulators, among many other applications.},
	number = {4},
	urldate = {2020-05-05},
	journal = {ACS Nano},
	author = {Butler, Sheneve Z. and Hollen, Shawna M. and Cao, Linyou and Cui, Yi and Gupta, Jay A. and Gutiérrez, Humberto R. and Heinz, Tony F. and Hong, Seung Sae and Huang, Jiaxing and Ismach, Ariel F. and Johnston-Halperin, Ezekiel and Kuno, Masaru and Plashnitsa, Vladimir V. and Robinson, Richard D. and Ruoff, Rodney S. and Salahuddin, Sayeef and Shan, Jie and Shi, Li and Spencer, Michael G. and Terrones, Mauricio and Windl, Wolfgang and Goldberger, Joshua E.},
	month = apr,
	year = {2013},
	note = {Publisher: American Chemical Society},
	pages = {2898--2926}
}

@article{ma_evidence_2012,
	title = {Evidence of the {Existence} of {Magnetism} in {Pristine} {VX2} {Monolayers} ({X} = {S}, {Se}) and {Their} {Strain}-{Induced} {Tunable} {Magnetic} {Properties}},
	volume = {6},
	issn = {1936-0851},
	url = {https://doi.org/10.1021/nn204667z},
	doi = {10.1021/nn204667z},
	abstract = {First-principles calculations are performed to study the electronic and magnetic properties of VX2 monolayers (X = S, Se). Our results unveil that VX2 monolayers exhibit exciting ferromagnetic behavior, offering evidence of the existence of magnetic behavior in pristine 2D monolayers. Furthermore, interestingly, both the magnetic moments and strength of magnetic coupling increase rapidly with increasing isotropic strain from −5\% to 5\% for VX2 monolayers. It is proposed that the strain-dependent magnetic moment is related to the strong ionic–covalent bonds, while both the ferromagnetism and the variation in strength of magnetic coupling with strain arise from the combined effects of both through-bond and through-space interactions. These findings suggest a new route to facilitate the design of nanoelectronic devices for complementing graphene.},
	number = {2},
	urldate = {2020-05-05},
	journal = {ACS Nano},
	author = {Ma, Yandong and Dai, Ying and Guo, Meng and Niu, Chengwang and Zhu, Yingtao and Huang, Baibiao},
	month = feb,
	year = {2012},
	note = {Publisher: American Chemical Society},
	pages = {1695--1701}
}

@article{zhang_direct_2013,
	title = {Direct observation of the transition from indirect to direct bandgap in atomically thin epitaxial {MoSe2}},
	volume = {9},
	doi = {10.1038/nnano.2013.277},
	abstract = {Quantum systems in confined geometries are host to novel physical phenomena. Examples include quantum Hall systems in semiconductors and Dirac electrons in graphene. Interest in such systems has also been intensified by the recent discovery of a large enhancement in photoluminescence quantum efficiency and a potential route to valleytronics in atomically thin layers of transition metal dichalcogenides, MX2 (M = Mo, W; X = S, Se, Te), which are closely related to the indirect-to-direct bandgap transition in monolayers. Here, we report the first direct observation of the transition from indirect to direct bandgap in monolayer samples by using angle-resolved photoemission spectroscopy on high-quality thin films of MoSe2 with variable thickness, grown by molecular beam epitaxy. The band structure measured experimentally indicates a stronger tendency of monolayer MoSe2 towards a direct bandgap, as well as a larger gap size, than theoretically predicted. Moreover, our finding of a significant spin-splitting of ∼180 meV at the valence band maximum of a monolayer MoSe2 film could expand its possible application to spintronic devices.},
	journal = {Nature nanotechnology},
	author = {Zhang, Yi and Chang, Tay-Rong and Zhou, Bo and Cui, Yong-Tao and Yan, Hao and Liu, Zhongkai and Schmitt, Felix and Lee, James and Moore, Rob and Chen, Yulin and Lin, Hsin and Jeng, Horng-Tay and Mo, Sung-Kwan and Hussain, Zahid and Bansil, Arun and Shen, Zhi-Xun},
	month = dec,
	year = {2013},
	file = {Full Text PDF:/Users/sweetie/Zotero/storage/IAG5PGZ7/Zhang et al. - 2013 - Direct observation of the transition from indirect.pdf:application/pdf}
}

@article{ugeda_characterization_2016,
	title = {Characterization of collective ground states in single-layer {NbSe} 2},
	volume = {12},
	issn = {1745-2473},
	url = {https://research.monash.edu/en/publications/characterization-of-collective-ground-states-in-single-layer-nbse},
	doi = {10.1038/nphys3527},
	language = {English},
	number = {1},
	urldate = {2020-05-05},
	journal = {Nature Physics},
	author = {Ugeda, Miguel M. and Bradley, Aaron J. and Zhang, Yi and Onishi, Seita and Chen, Yi and Ruan, Wei and Ojeda-Aristizabal, Claudia and Ryu, Hyejin and Edmonds, Mark T. and Tsai, Hsin-Zon and Riss, Alexander and Mo, Sung Kwan and Lee, Dunghai and Zettl, Alex and Hussain, Zahid and Shen, Zhi-Xun and Crommie, Michael F.},
	month = jan,
	year = {2016},
	note = {Publisher: Nature Publishing Group},
	pages = {92--97},
	file = {Snapshot:/Users/sweetie/Zotero/storage/6J9F8LYR/characterization-of-collective-ground-states-in-single-layer-nbse.html:text/html;Submitted Version:/Users/sweetie/Zotero/storage/UG3VS5ME/Ugeda et al. - 2016 - Characterization of collective ground states in si.pdf:application/pdf}
}

@article{wang_electronics_2012,
	title = {Electronics and optoelectronics of two-dimensional transition metal dichalcogenides},
	volume = {7},
	copyright = {2012 Nature Publishing Group, a division of Macmillan Publishers Limited. All Rights Reserved.},
	issn = {1748-3395},
	url = {https://www.nature.com/articles/nnano.2012.193},
	doi = {10.1038/nnano.2012.193},
	abstract = {Single-layer metal dichalcogenides are two-dimensional semiconductors that present strong potential for electronic and sensing applications complementary to that of graphene.},
	language = {en},
	number = {11},
	urldate = {2020-05-05},
	journal = {Nature Nanotechnology},
	author = {Wang, Qing Hua and Kalantar-Zadeh, Kourosh and Kis, Andras and Coleman, Jonathan N. and Strano, Michael S.},
	month = nov,
	year = {2012},
	note = {Number: 11
Publisher: Nature Publishing Group},
	pages = {699--712},
	file = {Full Text:/Users/sweetie/Zotero/storage/TMT59MMY/Wang et al. - 2012 - Electronics and optoelectronics of two-dimensional.pdf:application/pdf;Snapshot:/Users/sweetie/Zotero/storage/QK248I7V/nnano.2012.html:text/html}
}

@article{wang_electronics_2012-1,
	title = {Electronics and optoelectronics of two-dimensional transition metal dichalcogenides},
	volume = {7},
	issn = {1748-3387, 1748-3395},
	url = {http://www.nature.com/articles/nnano.2012.193},
	doi = {10.1038/nnano.2012.193},
	language = {en},
	number = {11},
	urldate = {2020-05-05},
	journal = {Nature Nanotechnology},
	author = {Wang, Qing Hua and Kalantar-Zadeh, Kourosh and Kis, Andras and Coleman, Jonathan N. and Strano, Michael S.},
	month = nov,
	year = {2012},
	pages = {699--712},
	file = {Wang et al. - 2012 - Electronics and optoelectronics of two-dimensional.pdf:/Users/sweetie/Zotero/storage/JX9MSEQL/Wang et al. - 2012 - Electronics and optoelectronics of two-dimensional.pdf:application/pdf}
}

@article{chhowalla_chemistry_2013,
	title = {The chemistry of two-dimensional layered transition metal dichalcogenide nanosheets},
	volume = {5},
	copyright = {2013 Nature Publishing Group, a division of Macmillan Publishers Limited. All Rights Reserved.},
	issn = {1755-4349},
	url = {https://www.nature.com/articles/nchem.1589},
	doi = {10.1038/nchem.1589},
	abstract = {Two-dimensional materials have recently garnered much interest in the scientific and technology communities. This Review describes how ultrathin transition metal dichalcogenides combine tunable structure and electronic properties, achieved through altering their composition, with versatile chemistry. This makes them attractive in various fields, for example as lithium-ion battery electrodes and electrocatalysts for the hydrogen evolution reaction.},
	language = {en},
	number = {4},
	urldate = {2020-05-05},
	journal = {Nature Chemistry},
	author = {Chhowalla, Manish and Shin, Hyeon Suk and Eda, Goki and Li, Lain-Jong and Loh, Kian Ping and Zhang, Hua},
	month = apr,
	year = {2013},
	note = {Number: 4
Publisher: Nature Publishing Group},
	pages = {263--275},
	file = {Full Text PDF:/Users/sweetie/Zotero/storage/KJ3M6LD7/Chhowalla et al. - 2013 - The chemistry of two-dimensional layered transitio.pdf:application/pdf;Snapshot:/Users/sweetie/Zotero/storage/KJNBGLWV/nchem.html:text/html}
}

@article{yang_long-lived_2015,
	title = {Long-lived nanosecond spin relaxation and spin coherence of electrons in monolayer {MoS} 2 and {WS} 2},
	volume = {11},
	copyright = {2015 Nature Publishing Group},
	issn = {1745-2481},
	url = {https://www.nature.com/articles/nphys3419},
	doi = {10.1038/nphys3419},
	abstract = {A range of semiconductors can host both spin and valley polarizations. Optical experiments on single layers of transition metal dichalcogenides now show that inter-valley scattering can accelerate spin relaxation.},
	language = {en},
	number = {10},
	urldate = {2020-05-05},
	journal = {Nature Physics},
	author = {Yang, Luyi and Sinitsyn, Nikolai A. and Chen, Weibing and Yuan, Jiangtan and Zhang, Jing and Lou, Jun and Crooker, Scott A.},
	month = oct,
	year = {2015},
	note = {Number: 10
Publisher: Nature Publishing Group},
	pages = {830--834},
	file = {Full Text PDF:/Users/sweetie/Zotero/storage/LG2QP354/Yang et al. - 2015 - Long-lived nanosecond spin relaxation and spin coh.pdf:application/pdf;Snapshot:/Users/sweetie/Zotero/storage/K6TKL46R/nphys3419.html:text/html}
}
%%%


@article{zhang_van_2019,
	title = {Van der {Waals} magnets: {Wonder} building blocks for two-dimensional spintronics?},
	volume = {1},
	issn = {2567-3165},
	shorttitle = {Van der {Waals} magnets},
	url = {https://onlinelibrary.wiley.com/doi/abs/10.1002/inf2.12048},
	doi = {10.1002/inf2.12048},
	abstract = {The unprecedented realization of two-dimensional (2D) van der Waals magnets excitingly extends the synergy between spintronics and 2D materials, started with graphene over the last decade. This article reviews the recent milestones in the development of 2D magnets and its derived heterostructures. In particular, a number of critical challenges centered around the scalability, ambient stability and Curie temperature of these atomically thin magnets are discussed. This mini-review also provides an outlook on what the future might hold for this integrated field of 2D spintronics, and assesses its potential in postsilicon electronics.},
	language = {en},
	number = {4},
	urldate = {2020-05-05},
	journal = {InfoMat},
	author = {Zhang, Wen and Wong, Ping Kwan Johnny and Zhu, Rui and Wee, Andrew T. S.},
	year = {2019},
	note = {\_eprint: https://onlinelibrary.wiley.com/doi/pdf/10.1002/inf2.12048},
	keywords = {spintronics, transition-metal chalcogenide, two-dimensional material, van der Waals magnets},
	pages = {479--495},
	file = {Snapshot:/Users/sweetie/Zotero/storage/Y2Y78C3P/inf2.html:text/html}
}
@article{mermin_absence_1966,
	title = {Absence of {Ferromagnetism} or {Antiferromagnetism} in {One}- or {Two}-{Dimensional} {Isotropic} {Heisenberg} {Models}},
	volume = {17},
	url = {https://link.aps.org/doi/10.1103/PhysRevLett.17.1133},
	doi = {10.1103/PhysRevLett.17.1133},
	abstract = {It is rigorously proved that at any nonzero temperature, a one- or two-dimensional isotropic spin-S Heisenberg model with finite-range exchange interaction can be neither ferromagnetic nor antiferromagnetic. The method of proof is capable of excluding a variety of types of ordering in one and two dimensions., This article appears in the following collection:},
	number = {22},
	urldate = {2020-05-05},
	journal = {Physical Review Letters},
	author = {Mermin, N. D. and Wagner, H.},
	month = nov,
	year = {1966},
	note = {Publisher: American Physical Society},
	pages = {1133--1136},
	file = {APS Snapshot:/Users/sweetie/Zotero/storage/GGBPGMFF/PhysRevLett.17.html:text/html}
}
@article{sethulakshmi_magnetism_2019,
	title = {Magnetism in two-dimensional materials beyond graphene},
	volume = {27},
	issn = {1369-7021},
	url = {http://www.sciencedirect.com/science/article/pii/S1369702118311696},
	doi = {10.1016/j.mattod.2019.03.015},
	abstract = {Magnetic materials enjoy an envious position in the area of data storage, electronics, and even in biomedical field. This review provides an overview of low-dimensional magnetism in graphene, h-BN, and carbon nitrides, which originates from defects like vacancy, adatom, doping, and dangling bonds. In transition metal dichalcogenides, a tunable magnetism comes from doping, strain, and vacancy/defects, and these materials offer spintronics, as well as photoelectronic potentials, since they have an additional degree of freedom called valley state (e.g. MoS2). Strain- and layer-dependent magnetic ordering has been observed in layered compounds like CrXTe3, CrI3, and trisulfides. The magnetism in 2D oxides like MoO3, Ni(OH)2,and perovskites are also interesting as they are potential candidates for next-generation devices having faster processing and large data storage capacity. Quasi 2D magnetism in MXene and in atomically thin materials supported on 3D materials will also be discussed. Finally, some of the challenges related to the control of defects and imperfections in 2D lattice, promising approaches to overcome them will be covered.},
	language = {en},
	urldate = {2020-05-05},
	journal = {Materials Today},
	author = {Sethulakshmi, N. and Mishra, Avanish and Ajayan, P. M. and Kawazoe, Yoshiyuki and Roy, Ajit K. and Singh, Abhishek K. and Tiwary, Chandra Sekhar},
	month = jul,
	year = {2019},
	pages = {107--122},
	file = {ScienceDirect Snapshot:/Users/sweetie/Zotero/storage/MQH55EN2/S1369702118311696.html:text/html}
}