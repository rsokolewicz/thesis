\pagenumbering{roman}
\frenchspacing
\raggedbottom

\thispagestyle{empty}
\vspace*{4em}
\begin{center}
\huge\spacedlowsmallcaps{Semiclassical dynamics of}\\
\huge\spacedlowsmallcaps{charge carriers in graphene}\\[1em]
\end{center}
\newpage

\thispagestyle{empty}
\null
\vfill

\setlength{\marginparwidth}{2em}
\setlength{\marginparsep}{0.75em}

%\marginnote[\copyright]{}
\noindent \copyright~ Koen Reijnders 2019


%SOMETHING FROM THE CLASSICTHESIS PACKAGE: remember to check the lastRoman page counter! It might miss 1 due to a cleardoublepage.
% TODO: Put the correct number of pages here!!!
\noindent Semiclassical dynamics of charge carriers in graphene\\
PhD thesis, Radboud University\\
% \pageref*{page:lastRoman} -> this prints ix, but we have x
x + 358 pages; illustrated, with bibliographic references \\ and popular summary in English and Dutch\\[2ex]
% {\scshape isbn} \quad 978\,$\cdot$\,94\,$\cdot$\,028\,$\cdot$\,1498\,$\cdot$\,9\\[2ex]
{\scshape isbn} \quad 978\,-\,94\,-\,028\,-\,1498\,-\,9\\[2ex]
% 978-94-028-1498-9
%Set in 9/12 pt Scala using \hologo{pdfLaTeX}\\
%Set in 9/12 pt using \hologo{pdfLaTeX}\\
%over design by Buro Brouns\\
Printed in the Netherlands by Ipskamp Printing\\[2ex]
%Available online as \textsc{cern}-\textsc{thesis}-2014-228\\[2ex]
%\textit{This work is part of the research programme of the \emph{Stichting voor Fundamenteel Onderzoek der Materie} \textsc{(fom)}, which is financially supported by the \emph{Nederlandse Organisatie voor Wetenschappelijk Onderzoek} \textsc{(nwo)}. The author was financially supported by \textsc{nwo}.}
This work was financially supported by \\ the \emph{Radboud University}, \\ the \emph{European Research Council} (\textsc{erc}) and \\ the \emph{Nederlandse Organisatie voor Wetenschappelijk Onderzoek} \textsc{(nwo)}


\newpage
% THIS IS THE REAL TITLE PAGE
\pdfbookmark[1]{Title}{titlepage}

\thispagestyle{empty}
\vspace*{4em}
\begin{center}
\huge\spacedlowsmallcaps{Semiclassical dynamics of}\\
\huge\spacedlowsmallcaps{charge carriers in graphene}\\[9em]
\normalsize


%Een wetenschappelijke proeve\\
%op het gebied van de Natuurwetenschappen,\\
%Wiskunde en Informatica\\[4em]



{\Large\scshape\textls[55]{proefschrift}}\\[2.5em]


ter verkrijging van de graad van doctor\\
aan de Radboud Universiteit Nijmegen\\
op gezag van de rector magnificus prof. dr. J.H.J.M. van Krieken,\\
volgens besluit van het college van decanen\\
in het openbaar te verdedigen op maandag 3 juni 2019\\
om 14:30 uur precies\\[1.5em]

door\\[1.5em]

{\Large\scshape\textls[55]{Koen Johannes Antonius Reijnders}}\\[1.5em]

geboren op 13 september 1986\\
te Nijmegen
\end{center}
\newpage
\thispagestyle{empty}

%Verso of title page: promotor, co-promotor, etc.
\noindent

    %\begin{tabular}{@{}ll}
    \begin{tabular}{ll}
    \spacedlowsmallcaps{promotor} & Prof. dr. M.I. Katsnelson\\
     & \\
    \spacedlowsmallcaps{manuscriptcommissie} \hspace*{0.5cm} & Prof. dr. N.P. Landsman (voorzitter)\\
                        & Prof. dr. J.S. Caux \\ %(Universiteit van Amsterdam) \\
                        & ~~~~(Universiteit van Amsterdam)\\
                        & Dr. V. Cheianov (Universiteit Leiden) \\
%                         &~~(Universiteit onbekend)\\
                        & Prof. dr. J. Schmalian (Karlsruher \\
                        & ~~~~Institut f\"ur Technologie, Duitsland) \\
%                         & ~~~~(Karlsruhe Institute of Technology) \\
                        & Prof. dr. H. Waalkens \\
                        & ~~~~(Rijksuniversiteit Groningen) \\
%                         & Prof. dr. A. AAAAA \\
%                         &~~(Universit\`a Roma Tre)\\
%                         & Dr. B. BBBBB \\
%                         &~~(Universiteit Leiden)\\
    \end{tabular}

\clearpage
% \thispagestyle{empty}
% \null


