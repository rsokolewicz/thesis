%*******************************************************
% Summary
%*******************************************************
\pdfbookmark[1]{Popular Summary}{Popular Summary}
\phantomsection
\manualmark
\markboth{\spacedlowsmallcaps{Popular Summary}}{\spacedlowsmallcaps{Popular Summary}}%
\addtocontents{toc}{}%
\addcontentsline{toc}{chapter}{\tocEntry{Popular Summary}}%
\chapter*{Popular Summary}%
% \bigskip
In this Thesis I studied the interaction between conducting electrons and the magnetic moments of atoms in a crystal. These electrons have properties similar to that of a refrigerator magnet, namely they also have a magnetic moment that we call \emph{spin}. Today's electronic devices make use of mostly the charge of electrons and not their spin. A new generation of electronics that makes use of the spin of these conducting electrons is often called \emph{spintronics} and could possibly lead to many benefits in today's technology. To name a few examples, our laptop and phone could become more energy efficient, be faster and be able to store more photos and videos. In the past two decades we have seen many technological advancements already in this field of spintronics such as the creation of electric currents where the spin of each electron is pointing in the same direction, or the creation of the first ferromagnet that is only a single atom thick. 

Although many experiments have shown that the magnetic moments of small crystals can be manipulated using electric currents in an efficient manner, and many phenomenological models exist, there only a few microscopic models out there that explain exactly what is going on. In this thesis, we consider a microscopic model, in which the conducting electrons can exchange their spin with the magnetic moments in the crystal through collisions with impurities in the crystal. This model is introduced in Chapter 1 as the “s--d-model” and in Chapter 2 we show the mathematical framework needed to calculate the dynamics and dissipation of magnetic moments under the influence of electric currents.

Our model is applied to two different crystals. First, in Chapter 3 we study topological semiconductors, which are characterized by a strong coupling between the spin of the conducting electrons and their momentum. By using a so-called “top gate” with a saw-tooth structure, it is possible to point the spin of conducting electrons either out of the crystal plane or into the crystal plane. Besides being a new way to manipulate the spin of conducting electrons, this effect could also be used to create a new kind of memory element.

In the remaining Chapters we study crystals characterized by a honeycomb structure, similar to graphene, where each atom has a magnetic moment that is anti-parallel with respect to their neighbors (see for example Figure \ref{fig:magnetic_phases} on page \pageref{fig:magnetic_phases}). In Chapter 4, we use a numerical method to calculate the torque that conductive electrons can exert on the magnetic moments in the crystal. We show that these torques become more efficient when the coupling between the spin of conducting electrons and the magnetic moment of the atoms in the crystal becomes asymmetrical. That is, if the coupling for all the magnetic moments that are pointing upwards, for example, is different from those that are pointing downwards.

In Chapter 5 we study the same crystal, but this time analytically. In addition to calculating torques, we also calculate how fast the magnetic moments of the crystal can dissipate under the influence of conducting electrons. We show that the different components of the magnetic moment not only dissipate to a different degree, but also that the component that points out of the crystal plane is preserved. In other words, if we can make a magnetic moment that points out of the crystal plane, we cannot undo it with the help of conducting electrons.

In the last chapter, we study the role of spin-orbit coupling on the dissipation of magnetic moments. Here we find three regimes with a different functional dependence of spin-orbit coupling on the dissipation. 