%*******************************************************
% Summary
%*******************************************************
\pdfbookmark[1]{Summary}{Summary}
\phantomsection
\manualmark
\markboth{\spacedlowsmallcaps{Summary}}{\spacedlowsmallcaps{Summary}}%
\addtocontents{toc}{}%
\addcontentsline{toc}{chapter}{\tocEntry{Summary}}%
\chapter*{Summary}%
% \bigskip
Every year new computers, phones, tables arrive to the market that are faster than before and can store more data than ever before. This ever increasing trend of making electronic devices better than before will soon reach a fundamental limit. 

We focus on materials, or combination of materials that can be considered "two-dimensional". This can either be (i) a crystal with the thickness of a single atom, (ii) a thin film, or (iii) two materials sandwiched together where the electrons are convinced to the interface. The collection of materials that are studied in this thesis possess certain electronic and magnetic properties. Firstly the materials should be able to conduct electrons and must have a linear relationship between their energy and their momentum and are called Dirac electrons. Secondly, 


can be grouped together and can be called Dirac ferromagnets and Dirac antiferromagnets. but still possesses relevant electronic and magnetic properties. 

By using the spin of the electron and not the charge as the information carrier, the new spin-based electronics, or "spintronics", could overcome all these problems and lead to a new generation of spin-based electronic devices for the information and communication technologies. Spintronics could potentially lead to lower power consumption, larger data processing speed and higher data storage capacity. 

The past two decades 
The past decade has shown many technological advancements, such has the exfoliation of the first one atom thick ferromagnet, CrI$_3$, the first  

As we slowly push electronics towards their  physical limits, more and more experiments show the promise of spintronics to save the day. Where in electronics only the charge of electrons is utilized, spintronics makes use of the fact that electrons carry a magnetic moment that we call spin as well. Devices based on spintronics are believed to be smaller, faster and more power-friendly than their electronic counter-part. Although some devices such as magnetic random access memory are available on the market today, much is still unknown about exact and relevant mechanisms involved. 

Moreover, two-dimensional magnetism reduces 
