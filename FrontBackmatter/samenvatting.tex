%*******************************************************
% Samenvatting
%*******************************************************
\pdfbookmark[1]{Populaire Samenvatting}{Populaire samenvatting}
\phantomsection
\manualmark
\markboth{\spacedlowsmallcaps{Populaire Samenvatting}}{\spacedlowsmallcaps{Populaire Samenvatting}}%
\addtocontents{toc}{}%
\addcontentsline{toc}{chapter}{\tocEntry{Populaire Samenvatting}}%
\chapter*{Populaire Samenvatting}%
% \bigskip
In dit proefschrift wordt de interactie tussen geleidende elektronen en de magnetische momenten van atomen in een kristal bestudeerd. Deze elektronen hebben eigenschappen vergelijkbaar met die van een koelkastmagneetje, namelijk ze hebben ook een magnetisch moment, die we spin noemen. In de hedendaagse elektronica, wordt voornamelijk gebruik gemaakt van de lading van de geleidende elektronen, en niet de spin. Een nieuwe generatie van elektronica die gebruikt maakt van de spin van deze geleidende elektronen wordt vaak \emph{spintronics} genoemd, en zou weleens vele voordelen kunnen bieden boven bestaande technologie. Om een paar voorbeelden te noemen, je laptop en telefoon zou met behulp van spintronica minder energie verbruiken, sneller zijn en meer foto's en video's op kunnen slaan. In de afgelopen twee decennia hebben we veel technologische vooruitgangen gezien op het gebied van spintronica, zoals het maken van elektrische stromen waar alle spins in dezelfde richting staan of het creëren van de eerste ferromagneet van 1 atoomlaag dik. 

Hoewel veel experimenten laten zien dat we magnetisme van kleine kristallen op efficiënte wijze kunnen manipuleren met behulp van elektrische stromen, en er veel fenomenologische modellen bestaan, bestaan er maar weinig microscopische modellen die uitleggen wat er nu precies gebeurd. In dit proefschrift beschouwen we een microscopisch model, waarbij de geleidende elektronen hun spin uit kunnen wisselen met de magnetische momenten in het kristal door middel van botsingen met oneffenheden in het kristal. Dit model wordt in Hoofdstuk 1 geïntroduceerd als het “s—d-model” en in Hoofdstuk 2 laten we het wiskundig kader zien waarmee de dynamica en dissipatie van magnetische momenten in kristallen uit kunnen rekenen onder invloed van elektrische stromen. 

Ons model wordt toegepast op twee verschillende kristallen. Allereerst bestuderen we in Hoofdstuk 3 topologische halfgeleiders, die gekenmerkt worden door een sterke koppeling tussen de spin van de geleidende elektronen en hun impuls. Door een zogenaamde “top gate” te gebruiken met een zaagtand structuur, is het mogelijk om de spin van geleidende elektronen dan wel uit het kristalvlak of in het kristalvlak te laten wijzen. Naast dat dit een nieuwe manier zou zijn om de spin van geleidende elektronen te manipuleren, zou dit effect ook gebruikt kunnen worden voor het maken van een nieuw soort geheugen element. 

In de overige Hoofdstukken bestuderen we kristallen die gekenmerkt worden door een honinggraatstructuur, vergelijkbaar met grafeen, waarbij ieder atoom een magnetisch moment heeft die anti-parallel staat ten opzichte van hun buren (zie bijvoorbeeld Figuur \ref{fig:magnetic_phases} op pagina \pageref{fig:magnetic_phases}). In Hoofdstuk 4 gebruiken we een numerieke methode om de krachtmomenten uit te rekenen die geleidende elektronen uit kunnen oefenen op de magnetische momenten in het kristal. We laten zien dat deze krachtmomenten efficiënter worden als de koppeling tussen de spin van geleidende elektronen en het magnetisch moment van de atomen in het kristal asymmetrisch wordt. Dat wil zeggen, als de koppeling voor alle magnetische momenten die bijvoorbeeld omhoog wijzen anders is dan die omlaag wijzen. 

In Hoofdstuk 5 bestuderen we hetzelfde kristal, maar dit keer analytisch. Naast het uitrekenen van krachtmomenten, rekenen we ook uit hoe snel de magnetische momenten van het kristal kunnen dissiperen onder invloed van geleidende elektronen. We laten zien dat de verschillende componenten van het magnetisch moment niet alleen met andere mate dissiperen, maar ook dat de component die uit het kristalvlak wijst, behouden blijft. Met andere woorden, als we een magnetisch moment maken die uit het kristalvlak wijst kunnen we deze niet ongedaan maken met behulp van geleidende elektronen. 

In het laatste hoofdstuk bestuderen we de rol van spin-baan koppeling op de dissipatie van magnetische momenten. We vinden hier drie regimes met een andere functionele afhankelijkheid van spin-baan koppeling op de dissipatie.   
