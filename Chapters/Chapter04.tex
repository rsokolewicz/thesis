%************************************************
\chapter{Highly anisotropic Gilbert Damping} % $\mathbb{ZNR}$
%************************************************
Giant Gilbert damping anisotropy is identified as a signature of strong Rashba spin-orbit coupling in a two-dimensional antiferromagnet on a honeycomb lattice. The phenomenon originates in spin-orbit induced splitting of conduction electron subbands that strongly suppresses certain spin-flip processes. As a result, the spin-orbit interaction is shown to support an undamped non-equilibrium dynamical mode that corresponds to an ultrafast in-plane N\'eel vector precession and a constant perpendicular-to-the-plane magnetization. The phenomenon is illustrated on the basis of a two dimensional $s$-$d$ like model. Spin-orbit torques and conductivity are also computed microscopically for this model. Unlike Gilbert damping these quantities are shown to reveal only a weak anisotropy that is limited to the semiconductor regime corresponding to the Fermi energy staying in a close vicinity of antiferromagnetic gap.

\vfill
Parts of this Chapter were published in Phys. Rev. \textbf{B} 101, 104403 (2020)\clearpage

\section{Introduction}
The N\'eel vector dynamics in an AFM is strongly affected by an interplay between different types of Gilbert dampings. Unlike in a simple single-domain ferromagnet with a single sublattice, the Gilbert damping in an AFM is generally different on different sublattices and includes spin pumping from one sublattice to another. A proper understanding of Gilbert damping is of key importance for addressing not only the mechanism of spin pumping but also domain wall motion, magnon lifetime, AFM resonance width and many other related phenomena \cite{PhysRevMaterials.1.061401, Kamra2018, Mahfouzi2018a, Yuan_2019, hals_phenomenology_2011}. It is also worth noting that spin pumping between two thin ferromagnetic layers with antiparallel magnetic orientations share many similarities with Gilbert damping in a bipartite AFM \cite{Heinrich2003,Tserkovnyak}. 

A conduction electron mechanism for Gilbert damping in collinear ferromagnet requires some spin-orbit interaction to be present. It is, therefore, commonly assumed that spin-orbit interaction of electrons naturally enhances the Gilbert damping. Contrary to this intuition, we show that Rashba spin-orbit coupling does generally suppress one of the Gilbert damping coefficients and leads to the appearance of undamped non-equilibrium N\'eel vector precession modes in the AFM. 
