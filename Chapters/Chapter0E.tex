\chapter{Calculation Details to Chapter 6}\label{app:D}
\section{Calculation of the anisotropy constant $K$}
The anisotropy constant is obtained from the grand potential energy $\Omega$ describing the conducting electrons. We express the grand potential density as
\begin{equation}
    \Omega = { -} \sum_{\varsigma=\pm}\frac{1}{\beta} \int \mathrm{d}\varepsilon\,g(\varepsilon) \nu_\varsigma(\varepsilon),
    \label{eq:b1}
\end{equation}
with
\begin{align}
    \nu_\varsigma(\varepsilon) = & \frac{1}{\pi} \tr_{\sigma,\Sigma} \int \frac{\mathrm{d}^2p}{(2\pi\hbar)^2}\,\im G^\text{R}_{\varsigma,\bb{p}}
    \label{eq:bnu}
\end{align}
where the trace runs over spin and sublattice indices $\varsigma=\pm$ denotes the valleys $\bb{K}$ and $\bb{K}'$,  $G^\text{R(A)}_{\varsigma,\bb{p}}$ are the clean retarded and advanced Green functions with momentum $\bb{p}$ (evaluated at zero disorder strength) in valley $\varsigma$, $\beta$ is the inverse temperature, and the function $g(\varepsilon)$ is given by
\begin{equation}
    g(\varepsilon) = \log \big(1+\exp[\beta(\mu-\varepsilon)]\big).
\end{equation}
We place the chemical potential in the upper band and the energy integration can be restricted to positive energies as the bottom two bands are filled and can only lead to a constant shift in $\Omega$ that we disregard. The evaluation of Eq.~(\ref{eq:bnu}) yields the following density of states
\begin{align}
    \nu_\tau(\varepsilon) 
    = \frac{1}{2\pi \hbar^2 v^2}& \begin{cases}
    0 & \text{for } 0 < \varepsilon < \varepsilon_2\\
    \varepsilon/2 +\lambda/4 & \text{for } \varepsilon_2 < \varepsilon < \varepsilon_1\\
    \varepsilon & \text{for } \varepsilon > \varepsilon_1
    \end{cases}
    % \label{eq:nu}
\end{align} 
where the energies $\varepsilon_{1,2}$ correspond to the extremal points top the bands (counted from above)
\begin{align}
    \varepsilon_{1,\varsigma} & = \frac{1}{2}\big(+\lambda+\sqrt{4\Delta^2+\lambda^2-4\varsigma\Delta\lambda n_z} \big),    \label{eq:elimits1}\\
    \varepsilon_{2,\varsigma} & = \frac{1}{2}\big(-\lambda+\sqrt{4\Delta^2+\lambda^2+4\varsigma\Delta\lambda n_z} \big)
    \label{eq:elimits2}
\end{align}

In the limit of zero temperature we can approximate Eq.~(\ref{eq:b1}) as
\begin{equation}
    \Omega = {-}\sum_{\varsigma=\pm}\frac{1}{\beta}\int_0^\infty \mathrm{d}\varepsilon\,(\mu-\varepsilon)\nu_\varsigma(\varepsilon)
    \label{eq:newomega}
\end{equation}
and by inserting Eq.~(\ref{eq:bnu}) in above's equation we find, 
\begin{multline}
    \Omega = -\sum_{\varsigma=\pm}\frac{(\varepsilon_{1,\varsigma}-\mu)^2(4\varepsilon_{1,\varsigma}-3\lambda+2\mu)+(\varepsilon_{2,\varsigma}-\mu)^2(4\varepsilon_{2,\varsigma}+3\lambda+2\mu)}{24\pi\hbar^2v^2}
\end{multline}
A careful analysis shows that the minimal energy corresponds to $n_z=\pm1$ so that the conducting electrons prefer an easy-axis magnetic anisotropy. By inserting Eqs.~(\ref{eq:elimits1}, \ref{eq:elimits2}) into the above equation we find
\begin{equation}
	\Omega = -\frac{(4\Delta^2-4n_z \Delta\lambda+\lambda^2)^{2/3}+(4\Delta^2+4n_z \Delta\lambda+\lambda^2)^{2/3}-24\Delta\mu+8\mu^3}{24\pi\hbar^2v^2}
\end{equation}
and by expanding in powers of $n_z^2$ around $n_z=\pm1$ we obtain 
\begin{equation}
    \Omega = -\frac{K}{2}n_z^2, \quad K= \frac{1}{2\pi\hbar^2v^2}\begin{cases}
    |\Delta^2\lambda| & \text{for } |\lambda/2\Delta| \geq 1 \\
    |\Delta\lambda^2| & \text{for } |\lambda/2\Delta| \leq 1
    \end{cases}.
\end{equation}