%************************************************
\chapter{Introduction} % $\mathbb{ZNR}$
%************************************************
\section{Introduction}
Ferromagnetic spintronics has greatly contributed to microelectronic technology in the last few decades \cite{bader_spintronics_2010, sinova_new_2012, bhatti_spintronics_2017}. One of the practical results was the development of magnetic memories with purely electronic write-in and read-out processes as an alternative to existing solid state drive technologies \cite{kent_new_2015, sato_two-terminal_2018}.

Ferromagnetic thin films have already entered commercial use in hard drives, magnetic field and rotation angle sensors and in similar devices  \cite{Parkin2003,Jogschies2015,Novoselov2019}, while keeping high promises for technologically competitive ultrafast memory elements \cite{Lau2016} and neuromorphic chips \cite{Fukami2016}. 

A gapless character of the spin-wave spectrum in isotropic Heisenberg magnets in two dimensions results in the homogeneity of magnetic ordering being destroyed by thermal fluctuations at any finite temperatures. In contrast, in van der Waals magnets, characterized by intrinsic magnetocrystalline anisotropy that stems from spin-orbit coupling \cite{Lado2017}, an ordered magnetic state can be retained down to a monolayer limit. Two-dimensional (2D) van der Waals magnets are currently experiencing a revived attention \cite{Gong2017,Herrero2017,Burch2018,Tokmachev2018,Gong2019,Novoselov2019,Cortie2019} driven by the prospects of gateable magnetism \cite{Huang2018,Shengwei2018,Wang2018,Deng2018}, a continuing search for Kitaev materials \cite{Nagler2019,Gordon2019} and Majorana fermions \cite{Livanas2019}, topologically driven phenomena \cite{Mokrousov2019} as well as various applications \cite{Herrero2017,Burch2018,Novoselov2019}. The trade-off between quantum confinement, nontrivial topology and long-range magnetic correlations determines their unique magnetoelectronic properties, in particular a tunable tunneling conductance \cite{Wang2018a} and magnetoresistance \cite{Song2018,Klein2018,Kim2018} depending on the number of layers in the sample, as well as long-distance magnon transport \cite{Xing2019}. 

It is widely known that spin-orbit interaction provides an efficient way to couple electronic and magnetic degrees of freedom. It is, therefore, no wonder that the largest torque on magnetization, which is also referred to as the spin-orbit torque, emerges in magnetic systems with strong spin-orbit interaction \cite{miron_current-driven_2010,haney_current_2013} as has been long anticipated \cite{dyakonov_current-induced_1971}. 

The spin-orbit coupling may be enhanced by confinement potentials in effectively two-dimensional systems consisting of conducting and magnetic layers. The in-plane current may efficiently drive domain walls or switch magnetic orientation in such structures with the help of spin-orbit torque \cite{awschalom2009trend,manchon_theory_2008,garate_influence_2009,manchon2009theory}, which is present even for uniform magnetization, or with the help of spin-transfer torque, which requires the presence of magnetization gradient (due to e.\,g. domain wall) \cite{slonczewski_current-driven_1996,berger_emission_1996,ralph_spin_2008,stiles_anatomy_2002}.

\section{Dynamics of ferromagnets}
\vfill
\clearpage