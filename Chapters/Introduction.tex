%************************************************
\chapter{Introduction}\label{ch:introduction}
%************************************************
\section{Spintronics and the limitations of every day electronics}
In todays society electronic devices are heavily incorporated in everyone's daily life. What is responsible for the ever increasing power of these devices is our better control of the electric charge. The transistor for example controls the flow of charge (turning it "on" and "off") by setting a gate voltage. The flow of charge however necessarily produces heat. Although a single unit in a microprocessor does not produce too much heat, fifty billion transistors.. It is predicted that in mere decades data centers and servers will demand roughly 50\% of the worlds energy demand just to cool the computers. 

The ever increasing industrial techniques make it possible to push electronics to their physical limit. As of writing ASML holds the record for most number of transistors built on a chip. Where in the early 70s a single computer chip could host up to a few thousands of transistors, it now approaches fifty billion. The physical limit ranges from the diffraction limit, which is overcome by using a laser with higher frequency, to the atomic limit. The average size of a transistor is ..., whereas the size of a bit in a magnetic hard drive is about. Another unwanted byproducts of each technological leap is the production of heat. 

Today's electronics revolves around the control of the electron's charge. By moving these charges around in electronic circuits, complex calculations can be made in mobile phones, computers,... As technology makes new advancements every year, we see that the size of electronic circuits gets smaller and smaller. A popular ``law'' used to show this advancement is named after Moore. It roughy states that the number of transistors that can be put unto a single chip doubles every two years. Where in the early 70s a single computer chip could host up to a few thousands of transistors, today the number approaches fifty billion.  

Another commercial area where spintronics could make a leapthrough is storage devices. Until a few years ago a magnetic hard-drive was the best storage medium. 

Electronics all revolve around the control of the charge of electrons. These electrons however also posses a magnetic moment called spin. The utilization of this spin degree of freedom, leads to the an emerging field called spintronics. Here the electrical control of spin-currents and magnetization is studied and it is believed that spintronic devices could potentially lead to more efficient data storage and faster reading/writing speeds than in conventional electronics. One of the main goals in this field is to discover how to efficiently convert charge into spin currents in devices without any magnetic contacts. 


% \section{Two dimensions}
% Many properties of a crystal are defined by its symmetries. At the surface of a three-dimensional crystal one of the translational symmetries is broken, leading to so-called surface states. These surface states are two-dimensional in nature and are in general very distinct from the states in the bulk. Similarly, electrons can also be confined between the interface of two crystals leading to similar two-dimensional states. Two dimensional systems have electronic and transport properties that are not found in three-dimensional crystals making them attractive alternatives to silicon-based electronics. Moreover, if the systems lack inversion symmetry it gives rise to Rashba spin-orbit coupling (more details below) which is a focus of the study in this thesis. 

% After the discovery of the first real two-dimensional crystal --- graphene --- many others were found. 

\section{Magnetism in 2D materials}

two D magnetism [Magnetism in two-dimensional materials beyond graphene]: single domain, increase in magnetic moment per atom, large directional anisotropy

The last couple of years have witnessed an exciting synergy between spintronics (Nobel prize in 2007) and two-dimensional (2D) materials (Nobel prize in 2010)
"A strong synergy between these two fields has been developed over the last couple of years, giving birth to anew field now referred to as 2D spintronics (Figure 1).The goal is to achieve “the best of both worlds” throughthe exploration and exploitation of spin functionality in2D materials. One intuitive advantage of 2D spintronics,over conventional spintronics using traditional magneticmaterials, is to provide a promising opportunity to pushthe relevant devices to the 2D limit that are also gate-tunableand chemical-tunable and mechanically flexible" 
[https://onlinelibrary.wiley.com/doi/epdf/10.1002/inf2.12048]

"A strong synergy between these two fields has been developed over the last couple of years, giving birth to anew field now referred to as 2D spintronics (Figure 1).The goal is to achieve “the best of both worlds” throughthe exploration and exploitation of spin functionality in2D materials. One intuitive advantage of 2D spintronics,over conventional spintronics using traditional magneticmaterials, is to provide a promising opportunity to pushthe relevant devices to the 2D limit that are also gate-tunableand chemical-tunable and mechanically flexible"

In this thesis, the effect of conducting electrons in two-dimensional ferro- and antiferromagnets is studied. More specifically, we focus on systems where the conducting electrons behave as Dirac fermions --- electrons whose energy is linear in the crystal momentum. A well known system whose electrons behave as Dirac fermions is graphene, the first two-dimensional crystal that was first discovered in 2004. A popular technique to produce a single layer of graphene is by exfoliating graphite. The same technique has led to the discovery of a multitude of new two-dimensional materials. Collectively, these materials are called van-der-Waals materials. In all compounds, the different layers are held together by weak van der Waals force. It has been predicted that the monolayer form of these materials are indeed stable 

These materials are often grouped in four distinct groups [spintronics book]: (i) graphene based, (ii) 2D chalcogenides, (iii) 2D halides and (iv) 2D oxides. The first category consists of materials that are a derivative of graphene --- such as fluorated graphene [] --- and materials that have a similar hexagonal crystal structure -- such as boron nitride []. The group of 2D chalcogenides consists of crystal containing at least one chalgenide atom (e.g. S, Se, Te) and have a layered structure similar to graphene. A commonly studied subgroup are the transition metal dichalcogenides (TMDs), whose crystal formula is given by MX$_2$. Here $M$ is a transition metal (e.g. Mo, W) and $X$ is a chalcogenide. The third group, 2D halides, contain crystals following a similar formula as the TMDs, e.g. MX$_2$ and MX$_3$, but with X being a halogen (e.g. Cl, Br, I). The crystal lattice of 2D chalcogenides and halides are generally hexagon shaped. The last group of 2D oxides ranges from simple crystals (e.g. ZnO) to more complex (e.g. Na$_2$Co$_2$TeO$_6$). Some examples of crystals in the last three groups are presented in Table~\ref{table:crystals}
\begin{table}[]
    \centering
    \begin{tabular}{l>{\raggedright}p{5cm}>{\raggedright\arraybackslash}p{5cm}}
    & \textbf{Ferromagnetic} & \textbf{Antiferromagnetic}\\
       (ii)  \textit{2D chalgenides}  & 
       %
       MoS$_2$, MoSe$_2$, VSe$_2$, MnSe$_2$, Fe$_3$GeTe$_2$, Cr$_2$Ge$_2$Te$_2$, Cr$_2$Si$_2$Te$_6$,CrGeTe$_3$
       %
       &
       %
       FePS$_3$, FePSe$_3$, MnPS$_3$,  MnPSe$_3$, NiPS$_3$, NiPSe$_3$, AgVP$_2$S$_6$, AgVP$_2$Se$_6$,  CrSe$_2$, CrTe$_3$, CrSiTe$_3$
       %
       \\\midrule
       (iii) \textit{2D halides}
       &
       %
       CrI$_3$ (single layer), CrBr$_3$, \\ GdI$_2$
       %
       &
       %
       CrI$_3$ (bi-layer), FeCl$_2$, CoCl$_2$,  NiCl$_2$, VCl$_2$, CrCl$_3$, FeCl$_3$,  FeBr$_2$,   MnBr$_2$, CoBr$_2$, VBr$_2$,  FeBr$_3$, FeI$_2$, VI$_2$, CrOCl,  CrOBr, CrSBr
       %
       \\\midrule
       (iv)  \textit{2D oxides}    & ZnO, MnO$_2$, $\delta$-FeOOH & Na$_2$Co$_2$TeO$_6$, Ni(OH)$_2$
    \end{tabular}
    \caption{Brief overview of various (anti)ferromagnetic van-der-Waals crystals [Magnetism in two-dimensional materials beyond graphene, Introduction to spintronics and 2D materials]. The first group of graphene based materials are non-magnetic and not shown in this Table. }
    \label{table:crystals}
\end{table}
The groups of two-dimensional chalgonides and halides are currently a popular topic of investigation as they provide access to many physical properties not found in other two-dimensional materials. Furthermore, the electronic properties, e.g. bandgap, of these crystals are highly tunable to doping, strain, and chemical composition. Large spin-orbit and spin-valley interactions lead to a much higher degree of spin-manipulation than found in heterostructures involving heavy metals. Furthermore, a large variety of magnetic phases are also found among these materials. 

Apart from intrinsic (anti)ferromagnetic crystals in Table~\ref{table:crystals}, there is also a large amount of crystals where (anti)ferromagnetism can be induced extrinsically. Techniques range from doping to magnetic proximity. By bringing a two-dimensional material in proximity with another material, one may manipulate its properties. For example spin-orbit interaction can be greatly enhanced by depositing a two-dimensional material on top of a heavy metal. In a similar fashion, (anti)ferromagnetism can be induced as well. 

Apart from van-der-waals structures, 2D magnetism can also be studied in interfaces and in surfaces of topological insulators. This case will be studied in Chapter~\ref{chap:diffusive} where we consider a bilayer consisting of a topological insulator (e.g. Bi$_2$Se$_3$) and an insulating magnet. 

chalgenides:
%%%%
Proc Natl Acad Sci U S A. 2005;102:10451-10453
wang QH, Kalantar-Zadeh K, Kis A, Coleman JN,
Strano MS. Electronics and optoelectronics of two-dimensional
transition metal dichalcogenides. Nat Nanotechnol. 2012;7:
699-712.

Chhowalla M, Shin HS, Eda G, Li W, Loh KP, Zhang H. The
chemistry of two-dimensional layered transition metal
dichalcogenide nanosheets. Nat Chem. 2013;5:263-275.
Radisavljevic B, Radenovic A, Brivio J, Giacometti V, Kis A.
Single-layer MOS-2 transistors. Nat Nanotechnol. 2011

Mak KF, Lee C, Hone J, Shan J, Heinz TF. Atomically thin
MOS : a new direct-gap semiconductor. Phys Rev Lett. 2010;105:
136805.

Xiao D, Liu GB, Feng WX, et al. Coupled spin and valley phys-
ics in monolayers of MoS2 and other group-VI dichalcogenides.
Phys Rev Lett.

Zeng HL, Dai JF, Yao W, Xiao D, Cui X. valley polarization in
monolayers by optical pumping. Nat Nanotechnol. 2012;7:
490-493.

Butler SZ, Holien SM, Cao LY, et al. Progress, challenges, and
opportunities in two-dimensional materials beyond graphene.
ACS Nano.

Ma YD, Dai Y, Guo M, Niu C, Zhu Y, Huang B. Evidence of the
existence of magnetism in pristine VX2 monolayers (X S, Se)
and their strain-induced tunable magnetic properties. ACS Nano.

Zhang Y, Chang TR, Zhou B, et al. Direct observation of the transition from indirect to direct bandgap in atomically thin epi- taxial MoS2. Nat Nanotechnol. 

Ugeda MM, Bradley AJ, Zhang Y, et al. Characterization of col- lective ground states in single-layer NbSe2. Nat Phys. 2016; 12: 92-97. 

Yang LY, Sinitsyn NA, Chen WB, et al. Long-lived nanosecond spin relaxation and spin coherence of electrons in monolayer MOS and WS2. Nat Phys.
%%%%


2D Dirac AFMs
massless
TaCoTe2 (https://arxiv.org/pdf/1910.07716.pdf)
https://journals.aps.org/prb/abstract/10.1103/PhysRevB.95.115138

Zr2Si %(https://pubs.rsc.org/en/content/articlelanding/2018/cp/c7cp08108a#!divAbstract)

BaFe2As2 and SrFe2As2 (https://journals.aps.org/prl/abstract/10.1103/PhysRevLett.119.096401)

https://journals.aps.org/prl/abstract/10.1103/PhysRevLett.118.106402
"Electric Control of Dirac Quasiparticles by Spin-Orbit Torque in an Antiferromagnet"

https://onlinelibrary.wiley.com/doi/full/10.1002/adma.201907565?casa_token=pRwIe09TghgAAAAA%3ASLIOXukGo4WJOwC5MNXQmnD4d345pyV76pEKAMWnadsciP7FBVInHvxQPcfZNNZb_0xQY9HZtMxs-_jexw
"Emergence of Nontrivial Low‐Energy Dirac Fermions in Antiferromagnetic EuCd2As2"

https://journals.aps.org/prb/abstract/10.1103/PhysRevB.101.161109
"Gapless Dirac surface states in the antiferromagnetic topological insulator 
MnBi2Te4"

honeycomb afm AB
https://link.springer.com/content/pdf/10.1140/epjb/e2020-100444-8.pdf
Inorganic materials such as Na2Co2TeO6 [15], BaCo2(AsO2)2 [16], are
examples of honeycomb lattice spin half antiferromagnets.

15. E. Lefancoise et al., Phys. Rev. B 94, 214416 (2016)
16. N. Martin, L.-P. Regnault, S. Klimko, J. Phys.: Conf. Ser.
340, 012012 (2012)

zigzag Na2Co2TeO6
https://journals.aps.org/prb/abstract/10.1103/PhysRevB.95.094424

Proximity Graphene:
- graphene + MnPSe3
"Quantum Anomalous Hall Effects in Graphene from Proximity-Induced Uniform and
Staggered Spin-Orbit and Exchange Coupling"
https://journals.aps.org/prl/abstract/10.1103/PhysRevLett.124.136403

- graphene + Cr2Ge2Te6 and WS2
"Purely interfacial and highly tunable spin-orbit torque operating field-effect transistor
in graphene doubly proximitized by two-dimensional magnet Cr2Ge2Te6 and WS2"
https://arxiv.org/pdf/1910.08072.pdf


tmdcs adatom doping magnetic proximity
nature nanotechnology 9 794 (2014)
prl 104 096804
nat phys 8 199
apl mater 4 032401
science 352 437 

intrinsic 2d magnets
41. N. Samarth, Condensed-matter physics: Magnetism in flatland, Nature 546, 216 (2017). 
42. B. Huang, G. Clark, E. Navarro-Moratalla et al., Layer-dependent ferromagnetism in a van der Waals crystal down to the monolayer limit, Nature 546, 270 (2017). 
43. C. Gong, L Li, Z. Li, et Discovery of intrinsic ferromagnetism in two-dimensional van der Waals crystals, Nature 546, 265 (2017). 
44. J.-U. Lee, S. Lee, J. H. Ryoo et al., Ising-type magnetic ordering in atomically thin FePS„ Nano Lett. 16, 7433 (2016). 
45. W. Xing, Y. Chen, P. M. Odenthal et al., Electric field effect in multilayer Cr2Ge2Te,: A ferromagnetic 2D material, 2D Mater. 4, 024009 (2017). 
46. M. Arai, R. Moriya, N. Yabuki, S. Masubuchi, K. Ueno, and T. Machida, Construction of van der Waals magnetic tunnel junction using ferromagnetic layered dichalcogenide, Appl. phys. Lett 107, 103107 (2015).

MoS2 spin valley coupling (vdW material)

magnetic properties graphene
https://www.sciencedirect.com/science/article/pii/B9780081021545000059

"Particularly, carbon-based
materials are of extreme interest since their structures are generally stable, simple, versatile
and easy to be modified, which results in easier theoretical magnetism prediction and more
likely spin induction."

" Experimentally, creating robust magnetic moments in
graphene remains very difficult. Approximately, the approaches can be roughly divided into
three categories: (1) the vacancy approach (creation of the magnetic moments on the basalplane sites by vacancy via ion irradiation), (2) the edge approach (creation of the edge
magnetic moments at the edge sites by edge-type defects), and (3) the sp3 approach"

vacancy: lieb's theorem

"showed the presence of localized electronic states at the zigzag edges (edge states), whereas
the armchair edge does not exhibit this feature [19,48]. The edge states of ZGNRs extend
along the edge direction and decay exponentially into the center of the ribbon. As a result
of the electron-electron interactions, magnetism appears in ZGNRs by ordering the spins
along the two ribbon edges with ferromagnetic coupling in the same edge, and
antiferromagnetic coupling between opposite edges [20,49]"

"There is now a large class of insulating 2D magnets with honeycomb lattice that can be placed on a heavy metal substrate
such as Pt or WTe2 to induce strong spin orbit interaction. The prominent examples include NiPS3, MnPS3, FePS3, NiPSe3,
MnPSe3, FePSe3, NbSe2 etc. There is also an itinerant 2D magnet Fe3GeTe2 that already posses a strong spin-orbit interaction
of Rashba type. The magnetization dynamics in these systems is actively studied experimentally. The honeycomb 2D magnets
can have both FM and AFM order and also support some non-collinear ground states."

"A strong synergy between these two fields has been developed over the last couple of years, giving birth to anew field now referred to as 2D spintronics (Figure 1).The goal is to achieve “the best of both worlds” throughthe exploration and exploitation of spin functionality in2D materials. One intuitive advantage of 2D spintronics,over conventional spintronics using traditional magneticmaterials, is to provide a promising opportunity to pushthe relevant devices to the 2D limit that are also gate-tunableand chemical-tunable and mechanically flexible"

"interface magnetic proximity"

"More examples beyond graphene have also been demonstrated in 2D transition-metal dichalcogenides ( TMDs) that have 2D layered structures similar to graphene,8,19-28including spin manipulation using spin-valley coupling in 2D-TMDs and the significantly larger spin-orbit torque than conventional heavy metals at ferromagnet/2D-TMDinterface."

Proc Natl Acad Sci U S A. 2005;102:10451-10453
wang QH, Kalantar-Zadeh K, Kis A, Coleman JN,
Strano MS. Electronics and optoelectronics of two-dimensional
transition metal dichalcogenides. Nat Nanotechnol. 2012;7:
699-712.
Chhowalla M, Shin HS, Eda G, Li W, Loh KP, Zhang H. The
chemistry of two-dimensional layered transition metal
dichalcogenide nanosheets. Nat Chem. 2013;5:263-275.
Radisavljevic B, Radenovic A, Brivio J, Giacometti V, Kis A.
Single-layer MOS-2 transistors. Nat Nanotechnol. 2011
Mak KF, Lee C, Hone J, Shan J, Heinz TF. Atomically thin
MOS : a new direct-gap semiconductor. Phys Rev Lett. 2010;105:
136805.
Xiao D, Liu GB, Feng WX, et al. Coupled spin and valley phys-
ics in monolayers of MoS2 and other group-VI dichalcogenides.
Phys Rev Lett.
Zeng HL, Dai JF, Yao W, Xiao D, Cui X. valley polarization in
monolayers by optical pumping. Nat Nanotechnol. 2012;7:
490-493.
Butler SZ, Holien SM, Cao LY, et al. Progress, challenges, and
opportunities in two-dimensional materials beyond graphene.
ACS Nano.
Ma YD, Dai Y, Guo M, Niu C, Zhu Y, Huang B. Evidence of the
existence of magnetism in pristine VX2 monolayers (X S, Se)
and their strain-induced tunable magnetic properties. ACS Nano.
Zhang Y, Chang TR, Zhou B, et al. Direct observation of the transition from indirect to direct bandgap in atomically thin epi- taxial MoS2. Nat Nanotechnol. Ugeda MM, Bradley AJ, Zhang Y, et al. Characterization of col- lective ground states in single-layer NbSe2. Nat Phys. 2016; 12: 92-97. Yang LY, Sinitsyn NA, Chen WB, et al. Long-lived nanosecond spin relaxation and spin coherence of electrons in monolayer MOS and WS2. Nat Phys.

"The first demonstration of 2D-vdW magnets is arguablythe defining moment of 2D spintronics.32,33Unlike most ofthe abovementioned milestone discoveries in graphene andgroup-VI 2D-TMDs where pseudo-spins are involved, 2D-vdW magnets exhibit active carrier spins, just like traditional3d ferromagnets"

Gong C, Li L, Li ZL, et al. Discovery of intrinsic ferromagnetismin two-dimensional van der Waals crystals. Nature. 2017;546:265-269

Huang B, Clark G, Navarro-Moratalla E, et al. Layer-dependentferromagnetism in a van der Waals crystal down to the mono-layer limit. Nature. 2017;546:270-273.

Castro Neto AH. Charge density wave, superconductivity, andanomalous metallic behavior in 2D transition metaldichalcogenides. Phys Rev Lett. 2001;86:4382-4385.

"The Mermin-Wagner theorem for many decades has servedas a “rule of thumb” for the understanding of 2D magnetism.55The theorem states that at any nonzero temperature, a long-range magnetic order, be it ferromagnetic or antiferromagnetic,cannot exist in a truly isotropic 2D system, because of its magnon excitation gap being incapable of resisting thermal agitations that collapse the spin ordering. It has been shown that even a small uniaxial magnetic anisotropy can open up a large magnon excitation gap, which in turn lifts the restric-tions imposed by the Mermin-Wagner theorem and results infinite Curie temperatures below which a 2D magnet can practically survive."

"Transition-metal phosphorus trichalcogenide MPX3(M = Mn, Fe, Co, Ni; X = S, Se) as an emerging group ofvdW-antiferromagnets has fueled much recent theoreticaland experimental efforts."
Joy PA, Vasudevan S. Magnetism in the layered transition-metalthiophosphates MPS3 (M = Mn, Fe, and Ni). Phys Rev B. 1992;46:5425-5433
Mayorga-Martinez CC, Sofer Z, Sedmidubský D, Huber Š,Eng AYS, Pumera M. Layered metal thiophosphite materials:magnetic, electrochemical, and electronic properties. ACS ApplMater Interfaces. 2017;9:12563-12573.

"For instance, antiferromagnetism in CoPS3and NiPS3is of theXY-type for which in the same layer, double-parallel ferromag-netic chains couple antiferromagnetically with each other"

"ThemagneticstructureofMnPS3consists of a magnetic ioncoupled antiferromagnetically with its nearest neighbors in thelayer, so that their moments are perpendicular to the layerplanes.128,130This is the isotropic Heisenberg-type. In FePS3,the magnetic easy axis also lies perpendicular to the layers.Within a layer, each Fe2+ion is ferromagnetically coupled withtwo of its nearest neighbors and antiferromagnetically with thethird one. However, unlike the XY-type spin orientation, the moments in one plane are antiferromagnetically coupled withneighboring planes, that is, an Ising-type antiferromagnet-ism.124,125,128,130,134,135"

124. Lee J-U, Lee S, Ryoo JH, et al. Ising-type magnetic ordering inatomically thin FePS3. Nano Lett. 2016;16:7433-7438.125. Wang X, Du K, Fredrik Liu YY, et al. Raman spectroscopy ofatomically thin two-dimensional magnetic iron phosphorus trisul-fide (FePS3) crystals. 2D Mater. 2016;3:031009.

128. Joy PA, Vasudevan S. Magnetism in the layered transition-metalthiophosphates MPS3 (M = Mn, Fe, and Ni). Phys Rev B. 1992;46:5425-5433.

130 Bernasconi M, Marra GL, Benedek G, et al. Lattice dynamics oflayered MP3(M = Mn, Fe, Ni, Zn; X = S, Se) compounds. PhysRev B. 1988;38:12089-12099.

134 Rule KC, Kennedy SJ, Goossens DJ, Mulders AM, Hicks TJ.Contrasting antiferromagnetic order between FePS3and MnPS3.Appl Phys A. 2002;74:s811-s813.

135 K-z D, Wang X-z, Liu Y, et al. Weak van der Waals stacking,wide-range band gap, and Raman study on ultrathin layers of metalphosphorus trichalcogenides. ACS Nano. 2016;10:1738-1743.

Fe3GeTe2 (https://advances.sciencemag.org/content/5/8/eaaw8904) on Pt and Pa
tunable curie temperature, conducting, large spin orbit.

\subsection{Spintronics in ferromagnets}
Ferromagnetic spintronics has greatly contributed to microelectronic technology in the last few decades \cite{bader_spintronics_2010, sinova_new_2012, bhatti_spintronics_2017}. One of the practical results was the development of magnetic memories with purely electronic write-in and read-out processes as an alternative to existing solid state drive technologies \cite{kent_new_2015, sato_two-terminal_2018}.

Ferromagnetic thin films have already entered commercial use in hard drives, magnetic field and rotation angle sensors and in similar devices \cite{Parkin2003,Jogschies2015,Novoselov2019}, while keeping high promises for technologically competitive ultrafast memory elements \cite{Lau2016} and neuromorphic chips \cite{Fukami2016}. 

The electronic control of ferromagnetic domains is described by spin-torques. The misalignment of the spins of conducting electrons and the magnetic moment of localized electrons lead to so-called spin-torques. These torques that are induced by electric currents are the key concept in spintronics and can be used to the reverse the direction of magnetization and to drive magnetic domain walls. We distinguish two types of current-induced spin torque: spin-transfer torque and spin-orbit torque. Spin-transfer torques generally vanish in uniform magnets and are proportional to gradients in magnetization. These torques arise from the transfer of spin angular momentum from a spin-current to the spin degree of freedom of a ferromagnetic region.  On the other hand, the spin-orbit torque is in general proportional to magnetization (and can therefore be found in uniform magnets) and is mediated by spin-orbit interactions. 

This last type of torque was first found in systems with a heavy metal layer [9], where the spin Hall
effect (SHE) [10–13] plays an important role [? ]. Due to its the strong thickness dependence, the SOT
produced by the SHE is limited to bulk materials [? ]. Later, more efficient charge-spin conversion was
found in heavy metal interfaces that support Rashba spin-orbit interaction [10, 14–16].
Recently spin- orbit torques were observed in bilayers consisting of a ferromagnet (FM) and a 3D
topological insulator (TI) [17, 18]. The 3D TI was proposed as a 3D extension of the Quantum Spin
Hall Insulator [19–22]. Its surface state consists of electrons that are described by a 2D Dirac
equation and are called Dirac fermions. Because the kinetic energy of the Dirac fermions is essentially
only spin-orbit interaction, spin and momentum are highly correlated. It is believed that this
spin-momentum locking – together with the topological properties of the TI – could lead to even
greater charge-spin conversion [23]. 

It is widely known that spin-orbit interaction provides an efficient way to couple electronic and magnetic degrees of freedom. It is, therefore, no wonder that the largest torque on magnetization, which is also referred to as the spin-orbit torque, emerges in magnetic systems with strong spin-orbit interaction \cite{miron_current-driven_2010,haney_current_2013} as has been long anticipated \cite{dyakonov_current-induced_1971}. 

The spin-orbit coupling may be enhanced by confinement potentials in effectively two-dimensional systems consisting of conducting and magnetic layers. The in-plane current may efficiently drive domain walls or switch magnetic orientation in such structures with the help of spin-orbit torque \cite{awschalom2009trend,manchon_theory_2008,garate_influence_2009,manchon2009theory}, which is present even for uniform magnetization, or with the help of spin-transfer torque, which requires the presence of magnetization gradient (due to e.\,g. domain wall) \cite{slonczewski_current-driven_1996,berger_emission_1996,ralph_spin_2008,stiles_anatomy_2002}. 

\subsection{Spintronics in antiferromagnets}
Antiferromagnets (AFMs) have received a great deal of attention due to their high temperature magnetic order in a large variety of materials. The study of
AFMs have uncovered many interesting properties. For example, the AFM has zero magnetic order, is
insensitive to external fields and contains no internal stray fields.
Furthermore, ultrafast switching of the antiferromagnetic order has been observed. In
spintronic devices these properties translate to storing magnetic data at high densities, fast
reading and writing, and, in addition to low energy consumption, making the AFM an ideal
candidate for further study.

In the past decade there has been a lot of focus on spintronic devices based on
ferromagnets (FMs), where they have proven their its potential. However,
AFMs are far more superior to FMs. For example, switching is mediated by the
exchange field as opposed to the much weaker anisotropy field in FMs, leading to
switching rates in the terahertz regime in contrast to those in the gigahertz regime found in
FMs. What’s more, experiments show that, in
AFMs, domain walls can travel with speeds of up to tens of km/s, whereas in FMs
the upper limit is only tens of cm/s.

The AFMs robustness against magnetic probing make it very difficult to manipulate their
magnetic state. Even though AFMs have been studied since the 1930s, it was only
realized in this decade that control of the antiferromagnetic order can be efficiently 
achieved by means of an electrical current, where the spin angular momentum of traveling
electrons are transferred to the magnetic angular momentum [REF?]. This effect is
known as the spin transfer torque (STT) and it was only very recently (in 2016) that the
technology has advanced far enough for this effect to be shown shown experimentally
[REF?]. Researchers were able to switch the antiferromagnetic order in CuMnAs thin-film,
which opened the way to spintronic devices based on AFMs.

It is important to know how electrical currents change the antiferromagnetic order and what
type of torques are responsible for this. The development of predictive and qualitative models is therefore important to in aiding further research in AFM- based spintronics.
Unfortunately, the dynamics of magnetization and spin in AFMs is far more complicated
than in FMs, so that few theoretical models exist for this system.

The first microscopic model describing STTs in AFMs was done constructed in 2006. where
the researchers numerically studied a one-dimensional spin chain with antiferromagnetic
order and made first predictions on switching induced by electric currents. Later
theoretical work consisted of more detailed numerical calculations and solving tight binding
models. Though both approaches give good results, none included disorder in their
calculation. Furthermore, these calculations do not illuminate which material
parameters are important. In 2017 a model was proposed that is based on spin
diffusion. Although this model does take the effect of disorder into account, it requires
electrons to diffuse into neighboring layers. 

An increasing demand for ever higher performance computation and ever faster big data analytics has sparked recently the interest to antiferromagnetic spintronics \cite{macdonald_antiferromagnetic_2011, gomonay_spintronics_2014, wadley_electrical_2016, jungwirth_antiferromagnetic_2016, baltz_antiferromagnetic_2018, jungwirth_multiple_2018, jungfleisch_perspectives_2018}, i.\,e. to the usage of much more subtle antiferromagnetic order parameter to store and process information. This idea is driven primarily by the expectation that antiferromagnetic materials may naturally allow for up to THz operation frequencies \cite{gomonay_high_2016, olejnik_terahertz_2018, jungwirth_multiple_2018} in sharp contrast to ferromagnets whose current-induced magnetization dynamics is fundamentally limited to GHz frequency range. 
  
The best efficiency in electric switching of magnetic domains is achieved in systems involving materials with at least partial spin-momentum locking \cite{fina_electric_2017} due to sufficiently strong spin-orbit interaction. The latter is responsible for the so-called spin-orbit torques on magnetization that are caused by sizable non-equilibrium spin polarization induced by an electron flow \cite{brataas_current-induced_2012, Hals2013, zelezny_relativistic_2014, freimuth_spin-orbit_2014, ghosh_spin-orbit_2017, smejkal_electric_2017, zelezny_spin_2018, zhou_strong_2018, manchon_current-induced_2019, moriyama_spin-orbit-torque_2018, li_manipulation_2019, chen_electric_2019, zhou_large_2019, zhou_fieldlike_2019, bodnar_writing_2018}.  

Recently, spin-orbit-torque-driven electric switching of the N\'eel vector orientation has been predicted \cite{zelezny_relativistic_2014} and discovered in non-centrosymmetric crystals such as CuMnAs \cite{wadley_electrical_2016, fina_electric_2017, zelezny_spin_2018, saidl_optical_2017} and Mn$_2$Au \cite{barthem_revealing_2013, jourdan_epitaxial_2015, bhattacharjee_neel_2018}. Even though many antiferromagnetic compounds are electric insulators \cite{pandey_doping_2017}, which limits the range of their potential applications, e.\,g., for spin injection \cite{tshitoyan_electrical_2015}, the materials like CuMnAs and Mn$_2$Au possess  semi-metal and metal properties, inheriting strong spin-orbit coupling and sufficiently high conductivity. These materials also give rise to collective mode excitations in THz range \cite{bhattacharjee_neel_2018}. 

Spin-orbit torque in antiferromagnets have been investigated theoretically using Kubo-Streda formula in the case of two-dimensional (2D) Rashba gas, as well as in tight-binding models of Mn$_2$Au \cite{zelezny_relativistic_2014, zelezny_spin-orbit_2017}. These pioneering works describe the spin-orbit torques based on their influence on the current-driven dynamics as predicted by Landau-Lifshitz-Gilbert equation. In particular, they proposed that only a staggered field-like torque and an non-staggered anti-damping torque can trigger current-driven antiferromagnetic THz switching and excitation in antiferromagnets (see e.\,g., \cite{fjaerbu_electrically_2017, cheng_terahertz_2016, khymyn_antiferromagnetic_2017}). This analysis served as a basis to further investigation of spin-orbit torques in heterostructures \cite{manchon_spin_2017, ghosh_spin-orbit_2017, ghosh_nonequilibrium_2019}. Furthermore, a symmetry group analysis has been developed to predict the form of these two components based on the magnetic point groups of the antiferromagnets \cite{zelezny_spin-orbit_2017, watanabe_symmetry_2018}. 
 
In the context of current-driven antiferromagnetic domain wall \cite{hals_phenomenology_2011} and skyrmion motion \cite{barker_static_2016, zhang_antiferromagnetic_2016}, the N\'eel vector dynamics has been modeled within the phenomenological treatment of the Landau-Lifshitz-Gilbert equation pioneered by Slonczewski \cite{slonczewski_current-driven_1996}. In this phenomenological approach the torques on magnetization derived above have been simply postulated \cite{gomonay_high_2016, shiino_antiferromagnetic_2016, akosa_theory_2018, tomasello_performance_2017}. 


Moreover, it has recently been suggested that current technology may have a lot to gain from antiferromagnet (AFM) materials. Indeed, manipulating AFM domains does not induce stray fields and has no fundamental speed limitations up to THz frequencies \cite{jungwirth_antiferromagnetic_2016}. 

Despite their ubiquitousness, AFM materials have, however, avoided much attention from technology due to an apparent lack of control over the AFM order parameter -- the N\'eel vector. Switching the N\'eel vector orientation by short electric pulses has been put forward only recently as the basis for AFM spintronics \cite{macdonald_antiferromagnetic_2011,gomonay_spintronics_2014,zelezny_relativistic_2014}. 
The proposed phenomenon has been soon observed in non-centrosymmetric crystals such as CuMnAs \cite{wadley_electrical_2016, fina_electric_2017, zelezny_spin_2018, saidl_optical_2017} and Mn$_2$Au \cite{barthem_revealing_2013, jourdan_epitaxial_2015, bhattacharjee_neel_2018}. It should be noted that in most cases AFMs are characterized by insulating type behavior \cite{pandey_doping_2017}, limiting the range of their potential applications, e.g., for spin injection \cite{tshitoyan_electrical_2015}. Interestingly, antiferromagnetic Mn$_2$Au possesses a typical metal properties, inheriting strong spin-orbit coupling and high conductivity, and is characterized by collective modes excitations in THz range \cite{bhattacharjee_neel_2018}.

Despite a lack of clarity concerning the microscopic mechanisms of the N\'eel vector switching, these experiments have been widely regarded as a breakthrough in the emerging field of THz spintronics \cite{bhattacharjee_neel_2018, gomonay_high_2016, olejnik_terahertz_2018, jungwirth_multiple_2018, wadley_electrical_2016, jungwirth_antiferromagnetic_2016, baltz_antiferromagnetic_2018, jungfleisch_perspectives_2018}. It has been suggested that current-induced N\'eel vector dynamics in an AFM is driven primarily by the so-called N\'eel spin-orbit torques \cite{brataas_current-induced_2012, Hals2013, zelezny_relativistic_2014, freimuth_spin-orbit_2014, ghosh_spin-orbit_2017, smejkal_electric_2017, zelezny_spin_2018, zhou_strong_2018, manchon_current-induced_2019, moriyama_spin-orbit-torque_2018, li_manipulation_2019, chen_electric_2019, zhou_fieldlike_2019, zhou_large_2019, bodnar_writing_2018}. The N\'eel spin-orbit torque originates in a non-equilibrium staggered polarization of conduction electrons on AFM sublattices \cite{zelezny_relativistic_2014, smejkal_electric_2017, zelezny_spin_2018, manchon_current-induced_2019}. Characteristic magnitude of the non-equilibrium staggered polarization and its relevance for the experiments with CuMnAs and Mn$_2$Au remain, however, debated. 


