%************************************************
\chapter{Introduction}\label{ch:introduction}
%************************************************
\section{Spintronics and the limitations of every day electronics}
In todays society electronic devices are heavily incorporated in everyone's daily life. What is responsible for the ever increasing power of these devices is our better control of the electric charge. The transistor for example controls the flow of charge (turning it "on" and "off") by setting a gate voltage. The flow of charge however necessarily produces heat. Although a single unit in a microprocessor does not produce too much heat, fifty billion transistors.. It is predicted that in mere decades data centers and servers will demand roughly 50\% of the worlds energy demand just to cool the computers. 

The ever increasing industrial techniques make it possible to push electronics to their physical limit. As of writing ASML holds the record for most number of transistors built on a chip. Where in the early 70s a single computer chip could host up to a few thousands of transistors, it now approaches fifty billion. The physical limit ranges from the diffraction limit, which is overcome by using a laser with higher frequency, to the atomic limit. The average size of a transistor is ..., whereas the size of a bit in a magnetic hard drive is about. Another unwanted byproducts of each technological leap is the production of heat. 

Today's electronics revolves around the control of the electron's charge. By moving these charges around in electronic circuits, complex calculations can be made in mobile phones, computers,... As technology makes new advancements every year, we see that the size of electronic circuits gets smaller and smaller. A popular ``law'' used to show this advancement is named after Moore. It roughy states that the number of transistors that can be put unto a single chip doubles every two years. Where in the early 70s a single computer chip could host up to a few thousands of transistors, today the number approaches fifty billion.  

Another commercial area where spintronics could make a leapthrough is storage devices. Until a few years ago a magnetic hard-drive was the best storage medium. 

Electronics all revolve around the control of the charge of electrons. These electrons however also posses a magnetic moment called spin. The utilization of this spin degree of freedom, leads to the an emerging field called spintronics. Here the electrical control of spin-currents and magnetization is studied and it is believed that spintronic devices could potentially lead to more efficient data storage and faster reading/writing speeds than in conventional electronics. One of the main goals in this field is to discover how to efficiently convert charge into spin currents in devices without any magnetic contacts. 


% \section{Two dimensions}
% Many properties of a crystal are defined by its symmetries. At the surface of a three-dimensional crystal one of the translational symmetries is broken, leading to so-called surface states. These surface states are two-dimensional in nature and are in general very distinct from the states in the bulk. Similarly, electrons can also be confined between the interface of two crystals leading to similar two-dimensional states. Two dimensional systems have electronic and transport properties that are not found in three-dimensional crystals making them attractive alternatives to silicon-based electronics. Moreover, if the systems lack inversion symmetry it gives rise to Rashba spin-orbit coupling (more details below) which is a focus of the study in this thesis. 

% After the discovery of the first real two-dimensional crystal --- graphene --- many others were found. 

\subsection{Spintronics in ferromagnets}
Ferromagnetic spintronics has greatly contributed to microelectronic technology in the last few decades \cite{bader_spintronics_2010, sinova_new_2012, bhatti_spintronics_2017}. One of the practical results was the development of magnetic memories with purely electronic write-in and read-out processes as an alternative to existing solid state drive technologies \cite{kent_new_2015, sato_two-terminal_2018}.

Ferromagnetic thin films have already entered commercial use in hard drives, magnetic field and rotation angle sensors and in similar devices \cite{Parkin2003,Jogschies2015,Novoselov2019}, while keeping high promises for technologically competitive ultrafast memory elements \cite{Lau2016} and neuromorphic chips \cite{Fukami2016}. 

The electronic control of ferromagnetic domains is described by spin-torques. The misalignment of the spins of conducting electrons and the magnetic moment of localized electrons lead to so-called spin-torques. These torques that are induced by electric currents are the key concept in spintronics and can be used to the reverse the direction of magnetization and to drive magnetic domain walls. We distinguish two types of current-induced spin torque: spin-transfer torque and spin-orbit torque. Spin-transfer torques generally vanish in uniform magnets and are proportional to gradients in magnetization. These torques arise from the transfer of spin angular momentum from a spin-current to the spin degree of freedom of a ferromagnetic region.  On the other hand, the spin-orbit torque is in general proportional to magnetization (and can therefore be found in uniform magnets) and is mediated by spin-orbit interactions. 

This last type of torque was first found in systems with a heavy metal layer [9], where the spin Hall
effect (SHE) [10–13] plays an important role [? ]. Due to its the strong thickness dependence, the SOT
produced by the SHE is limited to bulk materials [? ]. Later, more efficient charge-spin conversion was
found in heavy metal interfaces that support Rashba spin-orbit interaction [10, 14–16].
Recently spin- orbit torques were observed in bilayers consisting of a ferromagnet (FM) and a 3D
topological insulator (TI) [17, 18]. The 3D TI was proposed as a 3D extension of the Quantum Spin
Hall Insulator [19–22]. Its surface state consists of electrons that are described by a 2D Dirac
equation and are called Dirac fermions. Because the kinetic energy of the Dirac fermions is essentially
only spin-orbit interaction, spin and momentum are highly correlated. It is believed that this
spin-momentum locking – together with the topological properties of the TI – could lead to even
greater charge-spin conversion [23]. 

\subsection{Spintronics in antiferromagnets}
Antiferromagnets (AFMs) have received a great deal of attention due to their high temperature magnetic order in a large variety of materials. The study of
AFMs have uncovered many interesting properties. For example, the AFM has zero magnetic order, is
insensitive to external fields and contains no internal stray fields.
Furthermore, ultrafast switching of the antiferromagnetic order has been observed. In
spintronic devices these properties translate to storing magnetic data at high densities, fast
reading and writing, and, in addition to low energy consumption, making the AFM an ideal
candidate for further study.

In the past decade there has been a lot of focus on spintronic devices based on
ferromagnets (FMs), where they have proven their its potential. However,
AFMs are far more superior to FMs. For example, switching is mediated by the
exchange field as opposed to the much weaker anisotropy field in FMs, leading to
switching rates in the terahertz regime in contrast to those in the gigahertz regime found in
FMs. What’s more, experiments show that, in
AFMs, domain walls can travel with speeds of up to tens of km/s, whereas in FMs
the upper limit is only tens of cm/s.

The AFMs robustness against magnetic probing make it very difficult to manipulate their
magnetic state. Even though AFMs have been studied since the 1930s, it was only
realized in this decade that control of the antiferromagnetic order can be efficiently 
achieved by means of an electrical current, where the spin angular momentum of traveling
electrons are transferred to the magnetic angular momentum [REF?]. This effect is
known as the spin transfer torque (STT) and it was only very recently (in 2016) that the
technology has advanced far enough for this effect to be shown shown experimentally
[REF?]. Researchers were able to switch the antiferromagnetic order in CuMnAs thin-film,
which opened the way to spintronic devices based on AFMs.

It is important to know how electrical currents change the antiferromagnetic order and what
type of torques are responsible for this. The development of predictive and qualitative models is therefore important to in aiding further research in AFM- based spintronics.
Unfortunately, the dynamics of magnetization and spin in AFMs is far more complicated
than in FMs, so that few theoretical models exist for this system.

The first microscopic model describing STTs in AFMs was done constructed in 2006. where
the researchers numerically studied a one-dimensional spin chain with antiferromagnetic
order and made first predictions on switching induced by electric currents. Later
theoretical work consisted of more detailed numerical calculations and solving tight binding
models. Though both approaches give good results, none included disorder in their
calculation. Furthermore, these calculations do not illuminate which material
parameters are important. In 2017 a model was proposed that is based on spin
diffusion. Although this model does take the effect of disorder into account, it requires
electrons to diffuse into neighboring layers. 

\
