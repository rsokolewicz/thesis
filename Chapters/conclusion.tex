%%%%%%%%%%%%%%%%%%%%%%%%
\chapter{Conclusion}
%%%%%%%%%%%%%%%%%%%%%%%%

%%%
In conclusion, we consider magnetization dynamics in a model TI/FM system at a finite frequency $\omega$ and $\bb{q}$ vector. We identify novel diffusive anti-damping spin-orbit torque that is specific to TI/FM system.  Such a torque is absent in usual (non-topological) FM/metal systems, where the diffusive response of conduction electron spin density is always aligned with the magnetization direction of the FM. In contrast, the electrons at the TI surface gives rise to singular diffusive response of the conduction electron spin-density in the direction perpendicular to the TI surface, irrespective of the FM magnetization direction. Such a response leads to strong non-adiabatic anti-damping spin-orbit torque that has a diffusive nature. This response is specific for a system with an ultimate spin-momentum locking and gives rise to abnormal anti-damping diffusive torque that can be detected by performing spin-torque resonance measurements. We also show that, in realistic conditions, the anti-damping like diffusive torque may become orders of magnitude larger than the usual field like spin-orbit torque. We investigate the peculiar magnetizations dynamics induced by the diffusive torque at the frequency of the ferromagnet resonance. Our theory also predicts ultimate anisotropy of the Gilbert damping in the TI/FM system. In contrast, to the phenomenological approaches \cite{vanderBijl2012,Hals2013} our microscopic theory is formulated in terms of very few effective parameters. Our results are complementary to previous phenomenological studies of Dirac ferromagnets \cite{tserkovnyak_theory_2009,mahfouzi_spin-orbit_2012,katsnelson15,fischer_spin-torque_2016,yokoyama_theoretical_2010,yokoyama_current-induced_2011,siu_spin_2016,mahfouzi_antidamping_2016,soleimani_spin-orbit_2017,kurebayashi_microscopic_2017,chen_current-induced_2017,rodriguez-vega_giant_2016,qi_topological_2008,garate_inverse_2010,yokoyama_theoretical_2010,yokoyama_current-induced_2011,nomura_electric_2010,tserkovnyak_thin-film_2012-1,linder_improved_2014,tserkovnyak_spin_2015,ueda_topological_2012,liu_reading_2013,chang_nonequilibrium_2015,fischer_spin-torque_2016,mahfouzi_antidamping_2016,fujimoto_transport_2014,okuma_unconventional_2016}.

Motivated by recent experiments on the N\'eel vector switching we investigate microscopically the spin-orbit torques in an $s$--$d$-like model of a two-dimensional honeycomb antiferromagnet with Rashba spin-orbit coupling. We investigated the model with preserved and broken sublattice symmetry and distinguished metal and half-metal regimes for each of the model. Spin-orbit interaction in combination with on-site disorder potential and local exchange coupling between conduction and localized spins have been responsible for a microscopic mechanism of the angular momentum relaxation. We find identically vanishing anti-damping and N\'eel spin-orbit torques in the symmetric model in all regimes considered. As the result, the metal regime of the symmetric model is characterized by a particularly simple isotropic field-like spin-orbit torque, while the half-metal regime is characterized by anisotropic spin-orbit torques of the field-like symmetry.  Finite and anisotropic anti-damping torques, that crucially depend on disorder strength, are found in both metal and half-metal regimes of the asymmetric model. We also find non-equilibrium staggered polarization in the half-metal regime of the asymmetric model. This formally leads to a finite value of the N\'eel spin-orbit torque, which is, however, not a quantity of interest in that model. Overall, our results reveal the importance of two-dimensional electron momentum confinement for spin-orbit torque anisotropy. Largest values of spin-orbit torques are also associated with the half-metal regimes of conduction in both models. 

In this Chapter, we demonstrate that the presence of sufficiently strong spin-orbit coupling $\lambda\tau/\hslash \gg 1$ results in the ultimate anisotropy of the Gilbert damping tensor in the metal regime, $\ep_F\gg\Delta_{\text{sd}}+\lambda$.  One can trace the phenomenon to the spin-orbit induced splitting of Fermi surfaces that forbids scattering processes that are responsible for the relaxation of certain magnetization and N\'eel vector components. 

We also demonstrate that a finite in-plane projection $\bb{n}_\parallel$ of the N\'eel vector is responsible for a weak anisotropy of conductivity and spin-orbit torques for Fermi energies approaching the band edge, $\ep_F \sim \Delta_{\text{sd}}+\lambda$. This anisotropy is, however, absent in the metallic regime. 

Gilbert damping is, however, isotropic in the absence of spin-orbit interaction as it is required by symmetry considerations. Thus, we demonstrate that the onset of Rashba spin-orbit interaction in 2D or layered AFM systems leads to a giant anisotropy of Gilbert damping in the metallic regime. The physics of this phenomenon originates in spin-orbit induced splitting of the electron subbands that destroys a particular decay channel for magnetization and leads to undamped precession of the N\'eel vector. The phenomenon is based on the assumption that other Gilbert damping channels  (e.\,g. due to phonons) remain suppressed in the long magnon wavelength limit that we consider. The predicted giant Gilbert damping anisotropy may have a profound impact on the N\'eel vector dynamics in a variety of 2D and layered AFM materials. 

%%%%